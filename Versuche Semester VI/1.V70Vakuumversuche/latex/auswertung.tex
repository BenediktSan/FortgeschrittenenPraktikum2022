\newpage 
\section{Auswertung}

        \subsection{Fehlerrechnung}
        \noindent
        Die Fortpflanzung von Messungenauigkeiten für mehrere unabhängige Fehler wird durch die Gaußsche Fehlerfortpflanzung
        \begin{equation*}
        \Delta f = \sqrt{\sum_{i \, = \, 1}^{n} \, \left(\frac{\partial f}{\partial x_i} \, \Delta x_i\right)^2}
        \label{fehler}
        \end{equation*}
        beschrieben. Dabei gibt $\Delta x$ die Unsicherheit des arithmetischen Mittelwerts $\bar{x}$ einer Observablen $x$ an:
        
        \begin{equation*}
        \Delta x =  \sqrt{\frac{1}{n(n-1)} \sum_{i \, = \, 1}^{n} \, \left(\bar{x}- x_i\right)^2}.
        \end{equation*}

        \noindent
        Die Zahl $n$ gibt die Anzahl der unabhängigen Messungen an.\\\\
        Die Messwerte, die bei Messungen mit der Turbomolekularpumpe aufgenommen wurden, besitzen im Bereich $\SI{1e-8}{\milli\bar}$ bis $\SI{100}{\milli\bar}$ eine Ungenauigkeit von $30$\% \cite{V70}.
        Im Bereich von $\SI{100}{\milli\bar}$ bis $\SI{1000}{\milli\bar}$ sind es $50$ \%.\\
        Für die Messungen mit der Drehschieberpumpe sind es für Werte kleiner als $\SI{2e-3}{\milli\bar}$ ein Faktor $2$ vom Messwert \cite{V70}.
        Zusätzlich sind es von $\SI{2e-3}{\milli\bar}$ bis $\SI{10}{\milli\bar}$ $\pm\; \SI{120}{\milli\bar}$ 
        und von $\SI{10}{\milli\bar}$ bis $\SI{1200}{\milli\bar}$ $\pm \;\SI{3.6}{\milli\bar}$.\\\\
        Für die Fehlerfortpflanzung des logarithmischen Fits wird die aus der Formel \ref*{fehler} hergeleitete Formel 
        \begin{equation*}
          \sigma_{log} = \sqrt{\frac{\sigma_p^2}{(p-p_E)^2}+\frac{\sigma_{p0}^2}{(p_0-p_E)^2}+\sigma_{p_E}^2 \left(\frac{1}{p_0-p_E}-\frac{1}{p-p_E}\right)^2}
        \end{equation*}
        genutzt. Die Fehlerfortpflanzung der Saugleistung bestimmt sich zu
        \begin{equation*}
          \sigma_{S} = \sqrt{\left(\frac{m^2}{p_g^2}\right) \cdot \sigma_{V}^2 + \left(\frac{V^2}{p_g^2}\right) \cdot \sigma_m^2 + \left(\frac{m^2 V^2}{p_g^4}\right) \cdot \sigma_p^2} \quad .
        \end{equation*}
        Des Weiteren wird für die relative Abweichung berechneter Werte vom Theoriewert die Formel 
        \begin{equation*}
          \increment x = \frac{x - x_{theo}}{x_{theo}}
        \end{equation*}
        genutzt.

      \subsection{Volumen}
        Um das Saugvermögen der beiden Pumpen zu bestimmen wird das Volumen des Versuchsaufbaus benötigt.\\
        Für die Messungen, die für die Turbomolekularpumpe durchgeführt werden, wird ein kleineres Volumen benötigt.
        Dort wird nämlich nur das Volumen bis zur Turbomolekularpumpe evakuiert.\\
        Dieses Volumen beträgt $V_1 = \SI{33.0(33)}{\litre}$.\\
        Für die Messungen mit der Drehschieberpumpe wird zusätzlich noch das Volumen des Schlauchs genutzt. 
        Dieses beträgt $\SI{1.0(1)}{\litre}$, wodurch sich das Volumen $V_2 = \SI{34.0(34)}{\litre}$ ergibt.\\



      \subsection{Drehschieberpumpe}

        \noindent Für die Drehschieberpumpe werden im Folgenden beide Messverfahren ausgewertet. Einmal dass der Evakuierungsmessung und das der Leckratenmessung.\\
        Die Herstellerangabe des Saugvermögens der Drehschieberpumpe beträgt $ S_{theo} = \SI{1.1}{\litre\per\second}$\cite{V70}.

        \subsubsection{Evakuierungsmessung}

        \noindent Für diese Messung wurden über einen Zeitraum von $ T = \SI{600}{\second}$ alle $ \increment t = \SI{10}{\second}$
        der Druck im Versuchsaufbau hinter der deaktivierten Turbomolekularpumpe abgelesen. Dies wurde dreimal wiederholt.\\
        Die Messdaten sind in der Tabelle \ref{tab:dreh_p} im Anhang, inklusive der gemittelten Messwerte mit Fehler des Mittelwerts, zu finden.
        Als Enddruck wurde $\SI{0.021}{\milli\bar}$ aufgenommen. 
        %Da allerdings nicht gewartet wurde bis sich ein Enddruck eingestellt hat, musste der Wert aus dem 
        \\\\
        
        \noindent Für die Auswertung wurde ein linearer Fit der Form
        \begin{equation}
          y = m \cdot x + n
          \label{eqn:lin}
        \end{equation}
        auf die logarithmierten Messdaten, die der Form 
        \begin{equation*}
          \ln\left(\frac{p(t) - p_{end}}{p_0 - p_{end}}\right)
        \end{equation*}
        folgen, angewandt. Diese Gleichung ist dabei aus Gleichung \ref{eqn:druckkurve} hergeleitet.\\
        Allerdings ist das Saugvermögen der Pumpe nicht über alle Druckbereiche konstant, was zum Beispiel an unterschiedlichen Ströumungsarten bei unterschiedlichen Drücken liegt.\\
        Aus diesem Grund wurde der Fit für drei unterschiedliche Bereiche durchgeführt. Die Bereiche wurden aus der grafischen Darstellung abgeschätzt.\\
        Die Ergebnisse der Fits, die grafische Darstellung der Messdaten, inklusive Fehler, finden sich in Abbildung \ref{img:dreh_p}.
        Dabei sind auch die unterschiedlichen betrachteten Intervalle farblich hervorgehoben.\\
        \begin{figure}[h]
          \centering
          \includegraphics[width=0.7\textwidth]{build/plots/plot_dreh_p.pdf}
          \caption{Logarithmische Darstellung der Evakuierungsmesswerte von der Drehschieberpumpe.
          Dabei sind die unterschiedlichen genutzten Intervalle farblich hervorgehoben und alle dazugehörigen Ausgleichsgeraden eingezeichnet.}
          \label{img:dreh_p}
        \end{figure}

        \noindent Der erste Bereich geht vom Startdruck bis $\SI{1.8}{\milli\bar}$. Der zweite bis $\SI{0.44}{\milli\bar}$ und der dritte bis zum Ende.
        Aus den Fitparametern lässt sich über $ S = -mV $ das Saugvermögen bestimmen. Damit ergibt sich dann
        \begin{table}[H]
          \centering
          \small
          \label{tab:Saug_dreh_p}
          \begin{tabular}{l  c c c}
           \toprule
           {Bereich} & $\text{m} \mathbin{\scalebox{1.5} / } \si{\per\second}$ & $\text{n}$ & $\text{S} \mathbin{\scalebox{1.5} / } \si{\litre\per\second}$ \\
           \midrule
           $\SI{1}{\bar}$ bis $\SI{1.8}{\milli\bar}$ & -0.0310 \pm 0.00024 & -0.1173 \pm 0.00024 & 1.0532 \pm 0.10562 \\
           $\SI{1.8}{\milli\bar}$ bis $\SI{0.44}{\milli\bar}$ & -0.0110 \pm 0.00036 & -4.1260 \pm 0.0036 & 0.3727 \pm 0.03923\\
           $\SI{0.44}{\milli\bar}$ bis $\SI{0}{\milli\bar}$ & -0.0055 \pm 0.00005  & -5.9464 \pm 0.00005 & 0.1884 \pm 0.01891 \\
          \bottomrule
          \end{tabular}
          \caption{Parameter der Ausgleichsrechnungen und die Ergebnisse für das Saugvermögen.}
        \end{table} 


        \subsubsection{Leckratenmessung}

        \noindent Für die Leckratenmessung wurden über ein Ventil vier unterschiedliche Gleichgewichtsdrücke eingestellt
        und nach dem Verschließen des Ventils wieder Druckwerte aufgenommen.
        Die Messwerte für die Messung mit dem Gleichgewichtsdruck $p_G = \SI{0.4}{\milli\bar}$ sind in Tabelle \ref{tab:dreh_leck_1} zu finden.
        Die für $p_G = \SI{10}{\milli\bar}$ in Tabelle \ref{tab:dreh_leck_2}, $p_G = \SI{20}{\milli\bar}$ in Tabelle \ref{tab:dreh_leck_3} 
        und $p_G = \SI{80}{\milli\bar}$ in Tabelle \ref{tab:dreh_leck_4}.
        Sie wurden über einen Zeitraum von $ T = \SI{200}{\second}$ alle $ \increment t = \SI{10}{\second}$, hinter der deaktivierten Turbomolekularpumpe, aufgenommen.\\
        Grafisch aufbereitet, inklusive Ausgleichsgeraden und Abweichung, ist dies in den Abbildungen \ref{img:1}\subref{img:dreh_leck_1} bis  \ref{img:1}\subref{img:dreh_leck_4} zu finden.\\\\
        \begin{figure}[H]
            \begin{subfigure}{0.46\textwidth}
                    \centering
                    \includegraphics[width=\textwidth]{build/plots/dreh_04mbar.pdf}
                    \caption{Gleichgewichtsdruck $p_G = \SI{0.4}{\milli\bar}$.}
                    \label{img:dreh_leck_1}
            \end{subfigure}
            \hfill
            \begin{subfigure}{0.46\textwidth}
                    \centering
                    \includegraphics[width=\textwidth]{build/plots/dreh_10mbar.pdf}
                    \caption{Gleichgewichtsdruck $p_G = \SI{10}{\milli\bar}$.}
                    \label{img:dreh_leck_2}
            \end{subfigure}
            \hfill
            \begin{subfigure}{0.46\textwidth}
              \centering
              \includegraphics[width=\textwidth]{build/plots/dreh_40mbar.pdf}
              \caption{Gleichgewichtsdruck $p_G = \SI{20}{\milli\bar}$.}
              \label{img:dreh_leck_3}
            \end{subfigure}
            \hfill
            \begin{subfigure}{0.46\textwidth}
              \centering
              \includegraphics[width=\textwidth]{build/plots/dreh_80mbar.pdf}
              \caption{Gleichgewichtsdruck $p_G = \SI{80}{\milli\bar}$.}
              \label{img:dreh_leck_4}
            \end{subfigure}
            \caption{Die Messwerte der Leckratenmessung an der Drehschieberpumpe inklusive der dazugehörigen Ausgleichsgeraden.}
            \label{img:1}
        \end{figure}

        \noindent 
        Auf diese Messdaten wurden anschließend wieder lineare Ausgleichsgeraden, wie in Formel \ref{eqn:lin}, erstellt.
        Es wurden keine Datenpunkte exkludiert.\\
        Daraus lässt sich dann über den Zusammenhang $S = \frac{-V}{p_g}\cdot m$ das Saugvermögen bestimmen.\\
        Die Ergebnisse dieser Rechnungen sind in der folgenden Tabelle \ref{tab:erg_dreh_leck} aufgeführt.

        \begin{table}
          \begin{center}
            \begin{tabular}{c c c c}
              \toprule
              {$p_G \mathbin{\scalebox{1.5} / } \si{\milli\bar}$} & $\text{m} \mathbin{\scalebox{1.5} / } \si{\milli\bar\per\second}$ & 
              $\text{n} \mathbin{\scalebox{1.5} / } \si{\milli\bar}$  & $\text{S} \mathbin{\scalebox{1.5} / } \si{\litre\per\second}$ \\
              0.4                &    0.0084 \pm 0.0007    &     1.1197 \pm 0.0007         &  0.7310 \pm 0.0936 \\
              10                 &    0.3802 \pm 0.0042    &    14.5260  \pm 0.0042         &  1.2926 \pm 0.1300 \\
              40                 &    1.4150 \pm 0.0057   &     46.2518  \pm 0.0057         &  1.1988 \pm 0.1200 \\
              80                 &    2.7039 \pm 0.0407   &     87.8873 \pm 0.0407         &  1.1492 \pm 0.1162 \\ 
              \bottomrule 
            \end{tabular}
            \caption{Die Parameter der Regressionsgeraden und die daraus bestimmten Saugvermögen für die Drehschieberpumpe.}
            \label{tab:erg_dreh_leck}
          \end{center}
        \end{table}

        \subsection{Turbomolekularpumpe}

        \noindent Für die Turbomolekularpumpe wurden ebenfalls die Evakuierungsmessung und die Leckratenmessung durchgeführt.
        Die Auswertungen und Durchführungen der einzelnen Messungen sind weitestgehend identisch.
        Die Herstellerangabe des Saugvermögens der Turbomolekularpumpe beträgt $ S_{theo} = \SI{77}{\litre\per\second}$\cite{V70}.\\
        Außerdem ist bei den Auswertungen das veränderte Volumen zu beachten.

        \subsubsection{Evakuierungsmessung}

        \noindent
        Für die Evakuierungsmessung weicht die Auswertung leicht von der Drehschieberpumpe ab. 
        Hier wurden einmal Messwerte direkt an der Pumpe genommen und einmal am Vakuumkörper. 
        Die Messwerte, die an der Pumpe genommen wurden finden sich, inklusive gemittelter Werte, in Tabelle \ref{tab:turbo_p_pump} im Anhang. 
        Für die Messwerte am Vakuumkörper ist es Tabelle \ref{tab:turbo_p_vent}. \\
        Die Werte wurden über einen Zeitraum von $ T = \SI{200}{\second}$ alle $ \increment t = \SI{10}{\second}$ aufgenommen.\\
        Für beide Messungen wurde dafür als Endwert $p_{end} = \SI{1e-5}{\milli\bar}$ abgeschätzt.\\\\
        Die zu den Messwerten dazugehörigen Abbildungen sind Abbildung \ref{img:2}\subref{img:turbo_pump_p}, 
        für die Messungen an der Pumpe, und in Abbildung \ref{img:2}\subref{img:turbo_vent_p}, für den anderen Datensatz, zu finden.\\
        Der Unterschied zur Auswertung der Drehschieberpumpe ist, dass hier der Leitwert zusätzlich betrachtet wird, 
        weswegen ein zusätzlicher Datensatz an einer anderen Messstelle benötigt wird.
        Abgesehen davon ist die Auswertung analog zur Drehschieberpumpe.\\\\
        \begin{figure}[H]
          \begin{subfigure}{0.46\textwidth}
                  \centering
                  \includegraphics[width=\textwidth]{build/plots/plot_turbo_pump_p.pdf}
                  \caption{Die Daten die direkt oberhalb der Pumpe gemessen wurden.}
                  \label{img:turbo_pump_p}
          \end{subfigure}
          \hfill
          \begin{subfigure}{0.46\textwidth}
                  \centering
                  \includegraphics[width=\textwidth]{build/plots/plot_turbo_vent_p.pdf}
                  \caption{Die Daten die am Vakuumkörper gemessen wurden.}
                  \label{img:turbo_vent_p}
          \end{subfigure}
          \caption{Die Messwerte der Evakuierungsmessungen mit der Turbomolekularpumpe inklusive 
                    dazugehöriger Fits abgebildet. }
          \label{img:2}
      \end{figure}
        \noindent
        Auch hier wird das Saugvermögen wieder gesondert für drei Druckbereiche errechnet.
        Der erste Bereich geht von $\SI{167}{\milli\bar}$ bis $\SI{2.44}{\milli\bar}$. \\
        Der Zweite reicht dann bis $\SI{1.93}{\milli\bar}$ und der Dritte dann bis zum Ende der Messdaten.\\
        Für die Messungen am Vakuumkörper sind die Ergebnisse in Tabelle \ref{tab:Saug_turbo_vent} aufgelistet und für die an der Pumpe in Tabelle \ref{tab:Saug_turbo_pump}.

        \begin{table}[H]
          \centering
          \small
          \begin{tabular}{l c c c}
           \toprule
           {Bereich} & $\text{m} \mathbin{\scalebox{1.5} / } \si{\per\second}$ & $\text{n}$ & $\text{S} \mathbin{\scalebox{1.5} / } \si{\litre\per\second}$ \\
           \midrule
           $\SI{167}{\milli\bar}$ bis $\SI{2.44}{\milli\bar}$  & -0.1667 \pm 0.05531 & -0.9348 \pm 0.05531 & 5.4995 \pm 1.90637  \\
           $\SI{2.44}{\milli\bar}$ bis $\SI{1.93}{\milli\bar}$ &-0.0036 \pm 0.00027 & -5.0957 \pm 0.00027 & 0.1196 \pm 0.01482   \\
           $\SI{1.93}{\milli\bar}$ bis $\SI{0}{\milli\bar}$   &-0.0013 \pm 0.00003 & -5.2904 \pm 0.00003 & 0.0434 \pm 0.00447  \\
          \bottomrule
          \end{tabular}
          \caption{Parameter der Ausgleichsrechnungen für die Messungen am Vakuumkörper und die Ergebnisse für das Saugvermögen.}
          \label{tab:Saug_turbo_vent}
        \end{table} 

        \begin{table}[H]
          \centering
          \small
          \begin{tabular}{l c c c}
           \toprule
           {Bereich} & $\text{m} \mathbin{\scalebox{1.5} / } \si{\per\second}$ & $\text{n}$ & $\text{S} \mathbin{\scalebox{1.5} / } \si{\litre\per\second}$ \\
           \midrule
           $\SI{167}{\milli\bar}$ bis $\SI{2.44}{\milli\bar}$  & -0.1700 \pm 0.03266 & -0.1631 \pm 0.03266 & 5.6086 \pm 1.21505  \\
           $\SI{2.44}{\milli\bar}$ bis $\SI{1.93}{\milli\bar}$ & -0.0066 \pm 0.00039 1/s & -4.6240 \pm 0.00039 & 0.2182 \pm 0.02525   \\
           $\SI{1.93}{\milli\bar}$ bis $\SI{0}{\milli\bar}$    & -0.0026 \pm 0.00009  & -4.9640 \pm 0.00009 &  0.0866 \pm 0.00913 \\
          \bottomrule
          \end{tabular}
          \caption{Parameter der Ausgleichsrechnungen für die Messungen an der Pumpe und die Ergebnisse für das Saugvermögen.}
          \label{tab:Saug_turbo_pump}
        \end{table} 


        \noindent
        Um den Leitwert zu bestimmen wird der Zusammenhang
        \begin{align*}
          S_{eff} &= \frac{S_0 \cdot L }{S_0 + L}\\
          \iff L &= \frac{- S_{eff}\cdot S_0}{S_{eff} -S_0}
        \end{align*}
        genutzt.\\ Wenn dabei angenommen wird, dass das Saugvermögen am Vakuumkörper dem Idealwert entspricht und die Leistung an der Pumpe $S_{eff}$ entspricht, 
        lässt sich damit für die einzelnen Messbereiche der Leitwert bestimmen.\\
        Das Ergebnis dieser Rechnung ist in der folgenden Tabelle \ref{tab:leit} zu finden.
        \begin{table}[H]
          \centering
          \small
          \begin{tabular}{l c c}
           \toprule
           {Bereich} & $\text{L} \mathbin{\scalebox{1.5} / } \si{\litre\per\second}$  \\
           \midrule
           $\SI{167}{\milli\bar}$ bis $\SI{2.44}{\milli\bar}$  & -282.782 \pm 5911.4814  \\
           $\SI{2.44}{\milli\bar}$ bis $\SI{1.93}{\milli\bar}$ & -0.265 \pm 0.0815 \\
           $\SI{1.93}{\milli\bar}$ bis $\SI{0}{\milli\bar}$    & -0.087 \pm 0.0203 \\
          \bottomrule
          \end{tabular}
          \label{tab:leit}
        \end{table} 



        \subsubsection{Leckratenmessung}

        \noindent
        Hier werden die Rechnungen analog zur Drehschieberpumpe durchgeführt.\\
        Die Messdaten für die Messreihe mit Gleichgewichtsdruck $\SI{1e-4}{\milli\bar}$ sind in Tabelle \ref{tab:turbo_leck_1}, im Anhang, zu finden.
        Für den Gleichgewichtsdruck $\SI{2e-4}{\milli\bar}$ ist es Tabelle \ref{tab:turbo_leck_2}, für  $\SI{7e-5}{\milli\bar}$ Tabelle \ref{tab:turbo_leck_3}
        und für $\SI{5e-5}{\milli\bar}$ Tabelle \ref{tab:turbo_leck_4}.\\
        Der Zeitraum für diese Messung beträgt $ T = \SI{120}{\second}$. Das Messintervall ist wieder $ \increment t = \SI{10}{\second}$.
        Visualisiert sind die Messwerte, inklusive berechneter Fits, in den Abbildungen \ref{img:3}\subref{img:turbo_leck_1} bis \ref{img:3}\subref{img:turbo_leck_4}.
        \begin{figure}[H]
          \begin{subfigure}{0.46\textwidth}
                  \centering
                  \includegraphics[width=\textwidth]{build/plots/turbo_1e-4mbar.pdf}
                  \caption{Gleichgewichtsdruck $p_G = \SI{1e-4}{\milli\bar}$.}
                  \label{img:turbo_leck_1}
          \end{subfigure}
          \hfill
          \begin{subfigure}{0.46\textwidth}
                  \centering
                  \includegraphics[width=\textwidth]{build/plots/turbo_2e-4mbar.pdf}
                  \caption{Gleichgewichtsdruck $p_G = \SI{2e-4}{\milli\bar}$.}
                  \label{img:turbo_leck_2}
          \end{subfigure}
          \hfill
          \begin{subfigure}{0.46\textwidth}
            \centering
            \includegraphics[width=\textwidth]{build/plots/turbo_7e-4mbar.pdf}
            \caption{Gleichgewichtsdruck $p_G = \SI{7e-5}{\milli\bar}$.}
            \label{img:turbo_leck_3}
          \end{subfigure}
          \hfill
          \begin{subfigure}{0.46\textwidth}
            \centering
            \includegraphics[width=\textwidth]{build/plots/turbo_5e-4mbar.pdf}
            \caption{Gleichgewichtsdruck $p_G = \SI{5e-5}{\milli\bar}$.}
            \label{img:turbo_leck_4}
          \end{subfigure}
          \caption{Die Messwerte der Leckratenmessung an der Turbomolekularpumpe inklusive der dazugehörigen Fits. }
          \label{img:3}
      \end{figure}

      \noindent
      Auch hier wurde wieder eine lineare Ausgleichsrechnung durchgeführt, aus deren Ergebnis dann mittels $S = \frac{-V}{p_g}\cdot m$ das Saugvermögen bestimmt wird.\\
      Die Ergebnisse dieser Rechnungen sind in der folgenden Tabelle \ref{tab:erg_turbo_leck} aufgeführt.

      \begin{table}
        \begin{center}
          \begin{tabular}{c c c c}
            \toprule
            {$p_G \mathbin{\scalebox{1.5} / } \si{\micro\bar}$} & $\text{m} \mathbin{\scalebox{1.5} / } \si{\micro\bar\per\second}$ & 
            $\text{n} \mathbin{\scalebox{1.5} / } \si{\micro\bar}$  & $\text{S} \mathbin{\scalebox{1.5} / } \si{\litre\per\second}$ \\
            \midrule
            0.1                &   0.01568 \pm 0.000393    &    0.03797 \pm 0.000393         & 5.1067 \pm 0.52772 \\
            0.2                &   0.05111 \pm 0.002646    &    -0.27799 \pm 0.002646         & 8.3083 \pm 0.93574 \\
            0.07                &    0.00952 \pm 0.000221   &     0.06485 \pm 0.000221         & 4.4886 \pm 0.46088 \\
            0.05                &    0.00647 \pm 0.000094  &    0.07621 \pm 0.000094         &   4.2640 \pm 0.43101 \\ 
            \bottomrule 
          \end{tabular}
          \caption{Die Parameter der Regressionsgeraden und die daraus bestimmten Saugvermögen für die Turbomolekularpumpe.}
          \label{tab:erg_turbo_leck}
        \end{center}
      \end{table}























        
