\newpage
\section{Diskussion}

In der Auswertung folgt nun zunächst der Vergleich der experimentell bestimmten Absorptionskoeffizienten mit den 
Literaturwerten der unterschiedlichen Stoffe.

\begin{table}
    \centering
    \sisetup{table-format=2.1}
    \begin{tabular}{c c c c c}
    \toprule
    Eisen & Aluminium & Blei & Messing & Delrin\\
     \midrule 
    0.578 & 0.202 & 1.245 & 0.62 & 0.118\\
\bottomrule
\end{tabular}
\caption{Literaturwerte der Absorptionskoeffizienten in $\si{1\per\centi\metre}$ \cite{chemie} \cite{delrin}.}
\label{tab:lit}
\end{table}

\noindent Der Absorptionskoeffizient von Würfel 2 wurde zu $\mu_\t{Würfel 2} = \SI{0.059(0001)}{1\per\centi\metre}$ bestimmt.
Aus den zur Verfügung stehenden Materialien passt Delrin mit einer relativen Abweichung von 50\% am besten zu diesem Würfel.\\
\noindent
Für den homogenen Würfel 3 wurde ein Absorptionskoeffizienten von $\mu_\t{Würfel 3} = \SI{1.045(0004)}{1\per\centi\metre}$ berechnet.
Dieser passt mit einer relativen Abweichung von $\SI{16}{\percent}$ am besten zu Blei.\\
\noindent
Die Auswertung von Würfel 4 hat insgesamt sehr schlecht nur funktioniert. Die Absorptionskoeffizienten:
  \begin{table}[H]
    \centering
    \begin{tabular}{c S[table-format=3.4] @{${}\pm{}$} S[table-format=1.4] }
      \toprule
      {Würfel} & \multicolumn{2}{c}{$\mu \mathbin{/} \left(\si{\per\centi\metre}\right)$}  \\
      \midrule
      1 & -1.8506& 0.0088 \\
      2 & 0.3600 & 0.0080 \\
      3 & 4.0970 & 0.0119 \\
      4 & 1.2653 & 0.0064 \\
      5 & -0.4721& 0.0090 \\
      6 & 1.2207 & 0.0063 \\
      7 & -1.7298& 0.0089 \\
      8 & 0.2287 & 0.0079 \\
      9 & 4.1127 & 0.0118 \\
      \bottomrule
    \end{tabular}
  \end{table} 
\noindent
sind nicht sehr aussagekräftig, da einzelne Koeffizienten negativ sind. Somit sind die Ergebnisse nicht physikalisch und nicht interpretierbar.\\
Weiterhin sorgen die negativen Absorptionskoeffizienten in der Rechnung dafür, dass die Koeffizienten der anderen kleinen Blöcke in der Projektion größer werden als sie tatsächlich sind.
Gründe für diese schlechten Ergebnisse sind unterschiedliche systematische Fehler in diesem Versuch.
Primär ist der Strahlengang der Quelle wesentlich breiter als die zu untersuchenden Objekte, 
was zu unsauberen Projektionen führt, bei denen auch nicht betrachtete Elementarwürfel Beiträge liefern.
Dies tritt vor allem auf, wenn eine diagonale Projektion gewählt wird. 
Um dies, zu berücksichtigen müssten deutlich komplexere Berechnungen durchgeführt werden.\\
Weiterhin wird zur Berechnung der Absorptionskoeffizienten eine Matrix invertiert. 
Dies ist numerisch sehr instabil und führt schnell zu relevanten Fehlern.
Eine Änderung der Reihenfolge der Projektionen wurde probiert, hat jedoch dieselben schlechten Werte produziert und wurde daher in der Auswertung nicht aufgeführt.\\
Zur Bestimmung der Amplitude der einzelnen Projektionen wurden unterschiedliche Ansätze gewählt.\\
Zunächst wurde die Höhe des höchsten Peaks mit der Breite des Peaks bei der Hälfte das Maximums(FWHM) multipliziert, um einen Wert für die Fläche des Photopeaks zu bekommen.
Da sich der Verlauf des Photopeaks jedoch nicht großartig verändert hatte, sind die FWHM Werte zum Großteil identisch geblieben. 
Es wurde also hauptsächlich nur die Höhe des größten Bins ausgewertet. Dies hat wiederum große statistische Schwankungen, da sich die relevanten Energiewerte auf 2 Channel/Bins aufgeteilt haben.\\
Um dem entgegenzuwirken wurde die Net-Area als Maß für die Intensität gewählt. Dies ist ein vom Programm bestimmtes Integral über den Photopeak und hat zu besseren Ergebnissen geführt als die erste Methode.
Jedoch hat das Programm noch große Teile der Comptonkante dem Photopeak zugeordnet. 
Dies ist definitiv auch eine relevante Quelle für starke Abweichungen von den Literaturwerten.\\
Die Ergebnisse der Würfel aus einem Material passen, wenn der große systematische Fehler in Betracht gezogen wird, gut zu den Literaturangaben.
Für den zusammengesetzten Würfel gilt dies leider nicht, da dort sogar teilweise unphysikalische Ergebnisse erreicht wurden.
Da der Versuch aber einen großen systematischen Fehler besitzt, liefert er alles in allem vertretbare Ergebnisse.