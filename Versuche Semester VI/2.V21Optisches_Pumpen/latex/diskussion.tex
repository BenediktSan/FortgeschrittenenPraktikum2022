\section{Diskussion}

	Insgesamt lief der Versuch bis auf die leichten Probleme bei der Durchführung sehr gut, mit Ausnahme des Vertikalen Magnetfeldes wurden die Theoriewerte sehr gut getroffen.
	In dem Versuch wurde das vertikale Magnetfeld zu $\SI{35.25}{\micro\tesla}$ bestimmt, dies hat zu dem Theoriewert\cite{Erdmag} des Vertikalenmagnetfeldes in Deutschland von $\SI{44}{\micro\tesla}$ eine Abweichung von $\SI{19.89}{\percent}$.
	Dies lässt lässt sich damit erklären, dass das Magnetfeld innerhalb eines Gebädes mit Stahlbeton gemessen wurde.
	Außerdem konnter der optimale wert an dem der Effekt des Erdmagnetfeldes minimal ist, auch nur grob abgeschätzt werden.
	Die Bestimmung des Kernspins hat sehr gut geklappt, für die beiden Isoptope ergab sich:
	\begin{align*}
		&\text{Isotop 1:} & \text{berechnet: }\num{1.48(05)}\quad &\text{Theorie: }\num{1.5} &\text{Abweichung: }\SI{1}{\percent}  \\
		&\text{Isotop 2:} & \text{berechnet: }\num{2.54(05)}\quad &\text{Theorie: }\num{2.5} &\text{Abweichung: }\SI{1.7}{\percent}
	\end{align*}
	Hier und bereits in dem Auswetungsteil zu dem quadratischem Zeemann Effekt wurde bereits genutzt, dass das Isotop 1 mit einem Kernspin von \num{1.48}, das $^{87}\text{RB}$ ist. 
	Das Isotop 2 mit einem Kernspin von \num{2.54} ist wiederum das $^{85}{\text{RB}}$.
	Die Bestimmung der Anteile der Rubidium Isotope im Rubidium Gas\cite{Rubidium} hat leider nicht perfekt geklappt, für die Beiden Isotope ergab sich:
	\begin{align*}
		&^85\text{RB} & \text{berechnet: }\SI{67.857}{\percent} &\text{Theorie: }\SI{72.168}{\percent} &\text{Abweichung: }\SI{6.8}{\percent} \\
		&^87\text{RB} & \text{berechnet: }\SI{32.143}{\percent} &\text{Theorie: }\SI{27.835}{\percent} &\text{Abweichung: }\SI{15.47}{\percent} 
	\end{align*}
	Diese Fehler können daran liegen, dass die Berechnung nur mit einem Datenpunkt gemacht wurden, das genaue Ablesen der Peaktiefen aber fehlerhaft ist. 

