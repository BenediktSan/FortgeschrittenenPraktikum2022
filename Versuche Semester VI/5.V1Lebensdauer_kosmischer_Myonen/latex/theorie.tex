\section{Zielsetzung}
    Ziel dieses Versuches ist es die Lebensdauer kosmischer Myonen zu bestimmen und die verwendeten Schaltungen und Bauteile genauer zu verstehen.

    \section{Entstehung und Zerfall kosmischer Myonen}
        Myonen entstehen in der Atmosphäre in sogenannten Luftschauern, wenn hochenergetische kosmische Teilchen auf die Erdatmosphäre treffen.
        In unterschiedlichen Prozessen entstehen unter anderem geladene Pionen, welche wiederum in Myonen zerfallen:
        \begin{align}
            &\pi^+ \rightarrow \mu^+ + \nu_\mu && \text{und} && \pi^- \rightarrow \mu^- + \bar{\nu}_\mu \, .
        \end{align}
        Diese auf diesen Weg entstandenen Myonen sind so hochenergetisch, 
        dass sie nach dem Pionen Zerfall mit annähernd Lichtgeschwindigkeit in Richtung Erde fliegen.
        Nur aufgrund der relativistischen Geschwindigkeiten ist es den Myonen möglich auf der Erdoberfläche anzukommen. 
        Denn im Vergleich zum Elektron sind Myonen instabil und besitzen eine ca. 207-mal größere Ruhemasse. 
        Myonen zerfallen nach folgender Gleichung in Elektronen und Neutrinos:
        \begin{align}
            \mu^- \rightarrow e^- + \bar{\nu}_e + \nu_\mu && \text{und} &&  \mu^+ \rightarrow e^+ + \nu_e + \bar{\nu}_\mu \, . 
        \end{align} 

    \section{Lebensdauer instabiler Teilchen}

        Der Zerfall eines Elementarteilchens ist ein statistischer Prozess. Jedes Teilchen hat dieselbe Wahrscheinlichkeit zu Zerfallen.
        Die Wahrscheinlichkeit d$W$, dass ein Teilchen zerfällt ist also proportional zu der Zeit d$t$. Es folgt:
        \begin{equation}
            \t{d}W = \lambda\t{d}t \, .
        \end{equation}
        Daraus folgt für den Zerfall vieler Teilchen
        \begin{equation}
            \t{d}N = -N\t{d}W = -\lambda N \t{d}t \quad .
        \end{equation}
       Dabei ist d$N$ die Anzahl an Zerfällen in dem Zeitraum d$t$, wenn $N$ Teilchen betrachtet wurden.
       Integrieren dieser Gleichung ergibt das Zerfallsgesetz:
       \begin{equation}
           N(t) = N_0 \cdot \t{exp}(-\lambda t)\, .
       \end{equation}
       Hier ist $\lambda$ die Zerfallskonstante, $t$ die Zeit und $N_0$ die Gesamtzahl der betrachteten Teilchen. 
       Die Zeit, nachdem die noch übrig gebliebenen Teilchen $N(t)$ auf $\frac{N_0}{e}$ abgefallen sind, 
       wird als Lebensdauer $\tau$ eines Teilchens bezeichnet. Sie berechnet sich zu $\tau = \frac{1}{\lambda}$.





