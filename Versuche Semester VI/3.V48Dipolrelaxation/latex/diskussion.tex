\newpage
\section{Diskussion}

\noindent
Insgesamt lief der Versuch bis auf ein paar kleine Probleme bei der Durchführung sehr gut.
Einmal ist das Thermometer ausgefallen so, dass ein Messwert erst verspätet aufgenommen wurde.\\
Bei der Messung für die größere Heizrate musste eine Messung abgebrochen werden, da die Probe nicht genug magnetisiert war.
Außerdem wurde das Ziel von einer Heizrate von $b =\SI{1.5}{\kelvin\per\minute}$ mit $b_1 =\SI{1.36}{\kelvin\per\minute}$ etwas stärker verfehlt.
Dies sollte aber keinen besonderen Einfluss auf die Auswertung haben, da so immer noch genügend Messwerte existieren und der Abstand zwischen den beiden Heizraten so sogar noch größer ist.\\
Abgesehen davon gab es keine Probleme bei der Durchführung.\\\\

\noindent
Bei den Messwerten lässt sich, wie in den Abbildungen \ref{img:mitunter_15} und \ref{img:mitunter_2} ein Hauptmaximum, welches für die Auswertung genutzt wird, und ein Nebenmaximum für höhere Temperaturen erkennen.
Dies ist wie zu erwarten. Der Unterschied in der Größe des maximalen Stroms lässt sich über die unterschiedlichen Heizraten und damit unterschiedlichen Relaxationszeitenerklären.\\
Die relevanten bestimmten Größen sind in Tabelle \ref{tab:rel} aufgetragen. 
Dort ist auch zu erkennen, dass die einzelnen Aktivierungsenergien sich in derselben Größenordung befinden. 
Der Vergleich mit dem Literaturwert der Aktivierungsenergie, von $\ce{KBr}$ mit $W_{theo} = \SI{0.66}{\electronvolt}$\cite{lit}, 
lässt sich erkennen, dass die Werte sich zwar nicht immer im Rahmen ihrer Abweichungen mit der Literatur überschneiden, sich aber doch in der richtigen Größenordnung bewegen.
Die relative Abweichung, die über die Formel  
\begin{equation*}
    \increment x = \left| \frac{x - x_{theo}}{x_{theo}} \right|
\end{equation*}
bestimmt wird, zeigt auch das die bestimmten Werte insgesamt gut sind. Die Ergebnisse der Rechnungen sind auch in Tabelle \ref{tab:rel} zu finden.

\begin{table}[H]
    \centering
    \small
    \begin{tabular}{S [table-format=9.0]  c c l}
     \toprule
     {Auswertungsverfahren} & $\text{    }\tau_0 \mathbin{\scalebox{1.5} / } \si{\atto\second}$ & $\text{W} \mathbin{\scalebox{1.5} / } \si{\electronvolt}$ & $\text{relative Abweichung} \mathbin{\scalebox{1.5} / } \si{\percent}$ \\
     \midrule
     \text{Polarisation $\SI{1.36}{\kelvin}$ }  & $\SI{0.91 (072)}{}$     & 0.9442 \pm 0.0168 & 43.06   \\
     \text{Polarisation $\SI{1.94}{\kelvin}$ }  & $\SI{19.57 (7905)e3}{}$ & 0.7308 \pm 0.0861 & 10.73   \\
     \text{Integration $\SI{1.36}{\kelvin}$ }   & $\SI{1.06 (077)e6}{}$   & 0.6467 \pm 0.0153 & 2.02    \\
     \text{Integration $\SI{1.94}{\kelvin}$ }   & $\SI{2.27 (069)e9}{}$   & 0.4828 \pm 0.0642 & 26.85   \\
    \bottomrule
    \end{tabular}
    \caption{Die wichtigsten bestimmten Werte und die relative Abweichung der Aktivierungsenergien vom Literaturwert von $W= \SI{0.66}{\electronvolt}$.
    Für die relativen Abweichungen sind keine Fehler angegeben, da sie nicht für die ersten fünf Stellen auftreten.}
    \label{tab:rel}
  \end{table}

  \noindent Der große Unterschied der charakteristischen Relaxationszeiten $\tau_0$ lässt sich damit erklären, dass $\tau_0 \propto \frac{T_{max}^2}{W}\exp \left( \frac{W}{T_{max}} \right)$ gilt.
  Kleine Änderungen in $W$ und $T_{max}$ führen durch den exponentiellen Zusammenhang dann zu großen Änderungen in der charakteristischen Relaxationszeit.\\
  In der Abbildung \ref{img:tau} lässt sich auch erkennen, dass die einzelnen Kurtven für $\tau(T)$ halblogarithmisch aufgetragen ähnliche Verläufe haben, was dafür spricht,
  dass die Größenordnungsunterschiede gerechtfertgt sind.\\
  Insgesamt lässt sich also sagen, dass die Messungen gute Ergebnisse geliefert haben.
  