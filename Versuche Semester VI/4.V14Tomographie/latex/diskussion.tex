\newpage
\section{Diskussion}

In der Auswertung folgt nun zunächst der Vergleich der experimentell bestimmten Absorptionskoeffizienten mit den Literaturwerten unterschiedlicher Stoffe.

\begin{table}
    \centering
    \sisetup{table-format=2.1}
    \begin{tabular}{c c c c c}
    \toprule
    Eisen & Aluminium & Blei & Messing & Delrin\\
     \midrule 
    0.578 & 0.202 & 1.245 & 0.62 & 0.118\\
\bottomrule
\end{tabular}
\caption{Literaturwerte der Absorptionskoeffizienten in $\si{1\per\centi\metre}$ \cite{chemie} \cite{delrin}.}
\label{tab:lit}
\end{table}

\noindent Der Absorptionskoeffizienten von Würfel 2 wurde zu $\mu_\t{Würfel 2} = \SI{0.059(0001)}{1\per\centi\metre}$ bestimmt.
