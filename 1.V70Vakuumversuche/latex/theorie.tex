\section{Theorie}

    \subsection{Zielsetzung}

       
Vorbereitung: 

	1. Definition des Vakuums:
		-keine festen Objekte oder Flüssigkeiten,
                -extrem wenig Gas und extrem niedriger Gasdruck
                -"geringer Druck innerhalb eines Gefäßes als außerhalb (Atmosphärensruck)"
                -niedriger als 300mbar <- niedrister auf der Erdoberfläche vorkommende Atmosphärendruck

	2. Ideales Gas:
		-Vielzahl von Teilchen in ungeordneter Bewegung
		-Wechselwirkung nur durch harte, elastische Stöße

	   Boyle-Mariottesches Gesetz:
		-konstante Teilchenzahl N, ideales Gas, konstante Temperatur, -> Druck oder Volumenänderung => isotherme Zustandsänderung
		 \-> Volumen V ist anti proportional zum Druck p => P * V = const 

	   Zustandsgleichung für idealee Gase:
		- p * V = N * kB * T

	   erwateter Zusammenhang zwischen Druck und Zeit für Evakuierungskurve und Leckratenmessung:
	   	- Die "Evakuierungsrate" nimmt exponentiell mit der Zeit ab somit steigt der Druck logarithmisch
		- Der durch ein Leck Druck nimmt exponentiell ab

	3. Druck:
		-Druck ist die Kraft auf eine Fläche

	   Partialdruck:
	   	-Der Druck der in einem Gasgemisch durch eine einzelne/oder mehrere Komponente entsteht
		-Setzt sich zum Gesamtdruck additativ zusammen

	   Druckeinheiten:
	   	-Technischer Atmospährendruck atm = kp/cm^2 = 98,0665 kPa 
		-Bar bar = 100kPa about equals 1at
		-Torr, Druck von eimem Millimeter Quecksilber mmHg= 1/760 atm
		-1 Meter Wassersäule mWS = 0,1atm = 9.8kPa

	   Teilchenzahldichte:
	   	- Anzahl an Teilchen pro Volumen, n oder C

	   Teilchengeschwindigkeit:

	   mittlere freie Weglänge:
	   	- Durchschnittliche Länge die ein Teilchen zwischen 2 Kollisionen fliegt

	4. Laminare Strömung:
		-Eine Strömung ohne sichtbare Turbulenzen/Verwirblungen
		-Das Fluid strömt in Schichten

	   molekulare Strömung:
		- die mittlere freie Weglänge ist deutlich größer als der Durchmesser der Strömung
		- Konstanter Fluss bei gleichem Druck

	   Leitwert:
		-Maß des Widerstandes beim Fluss eines Fluides durch ein Kabel analog zur elektrischen Leitfähigkeit? 

	5. Gasstrom: 
		-Fluss an Gass also Materie sehr geringer Dichte?

	   Saugleistung:
		-dpV/dt, das auf die Stoffmenge bezogene Durchlassvermögen

	   Saugvermögen:
	   	-Volumen pro Zeit bei Umgebungsdruck = 1bar und 20 Grad Celsius

	   effektives Saugvermögen einer Vakuumpumpe:
	   	-Das effektive Saugvermögen ist das Saugvermögen um einen Gewissen Faktor verringert, dieser berechnet sich aus dem Verhältnis des Druckes am Vakuumbehälter und dem Ansaugstutzen oder durch:
			S(eff) = S/(1 + S/L) mit dem Leitwert L

	   Leitwert eines Rohres:
		- (pi * d^4)/(256 * eta *l) * (p1+p2) Laminar: 

	6. Adsorption:

	   Absorption:

	   Desorption:

	   Diffusion:

	   "virtuelles" Leck:

	7. Methoden der Vakuumerzeugung:
		Funktionsweise von:

		Drehschieberpumpe:

		Turbomolekularpumpe:

	   Methoden der Vakuummessung:
	   	Funktionsweise von:

		Wärmeleitsungs-Vakuummeter:

		Ionisations-Vakuummeter:
			Kaltkathode:

			Glühkathode:


	---> Auf Druckbereiche beachten!
	---> Vor-, und Nachteile beachten!


