\section{Theorie}

    \subsection{Zielsetzung}

       
Vorbereitung: \\
\\
	1. Definition des Vakuums:\\
		-keine festen Objekte oder Flüssigkeiten,\\
                -extrem wenig Gas und extrem niedriger Gasdruck\\
                -"geringer Druck innerhalb eines Gefäßes als außerhalb (Atmosphärensruck)"\\
                -niedriger als 300mbar <- niedrister auf der Erdoberfläche vorkommende Atmosphärendruck\\
\\
	2. Ideales Gas:\\
		-Vielzahl von Teilchen in ungeordneter Bewegung\\
		-Wechselwirkung nur durch harte, elastische Stöße\\
\\
	   Boyle-Mariottesches Gesetz:\\
		-konstante Teilchenzahl N, ideales Gas, konstante Temperatur, -> Druck oder Volumenänderung => isotherme Zustandsänderung\\
		 \-> Volumen V ist anti proportional zum Druck p => P * V = const \\
\\
	   Zustandsgleichung für idealee Gase:\\
		- p * V = N * kB * T\\
\\
	   erwateter Zusammenhang zwischen Druck und Zeit für Evakuierungskurve und Leckratenmessung:\\
	   	- Die "Evakuierungsrate" nimmt exponentiell mit der Zeit ab somit steigt der Druck logarithmisch\\
		- Der durch ein Leck Druck nimmt exponentiell ab\\
\\
	3. Druck:\\
		-Druck ist die Kraft auf eine Fläche\\
\\
	   Partialdruck:\\
	   	-Der Druck der in einem Gasgemisch durch eine einzelne/oder mehrere Komponente entsteht\\
		-Setzt sich zum Gesamtdruck additativ zusammen\\
\\
	   Druckeinheiten:\\
	   	-Technischer Atmospährendruck atm = kp/$cm^2$ = 98,0665 kPa \\
		-Bar bar = 100kPa about equals 1at\\
		-Torr, Druck von eimem Millimeter Quecksilber mmHg= 1/760 atm\\
		-1 Meter Wassersäule mWS = 0,1atm = 9.8kPa\\
\\
	   Teilchenzahldichte:\\
	   	- Anzahl an Teilchen pro Volumen, n oder C\\
\\
	   Teilchengeschwindigkeit:\\
		-Die durchschnittliche Geschwindigkeit von Teilchen?\\
\\
	   mittlere freie Weglänge:\\
	   	- Durchschnittliche Länge die ein Teilchen zwischen 2 Kollisionen fliegt\\
\\
	4. Laminare Strömung:\\
		-Eine Strömung ohne sichtbare Turbulenzen/Verwirblungen\\
		-Das Fluid strömt in Schichten\\
\\
	   molekulare Strömung:\\
		- die mittlere freie Weglänge ist deutlich größer als der Durchmesser der Strömung\\
		- Konstanter Fluss bei gleichem Druck\\
\\
	   Leitwert:\\
		-Maß des Widerstandes beim Fluss eines Fluides durch ein Kabel analog zur elektrischen Leitfähigkeit?\\ 
\\
	5. Gasstrom: \\
		-Fluss an Gass also Materie sehr geringer Dichte?\\
\\
	   Saugleistung:\\
		-dpV/dt, das auf die Stoffmenge bezogene Durchlassvermögen\\
\\
	   Saugvermögen:\\
	   	-Volumen pro Zeit bei Umgebungsdruck = 1bar und 20 Grad Celsius\\
\\
	   effektives Saugvermögen einer Vakuumpumpe:\\
	   	-Das effektive Saugvermögen ist das Saugvermögen um einen Gewissen Faktor verringert, dieser berechnet sich aus dem Verhältnis des Druckes am Vakuumbehälter und dem Ansaugstutzen oder durch:\\
			S(eff) = S/(1 + S/L) mit dem Leitwert L\\
\\
	   Leitwert eines Rohres:\\
		- (pi * $d^4$)/(256 * eta *l) * (p1+p2) Laminar:\\
\\
	6. Adsorption:\\
		-Wenn Materie sich an der Oberfäche von Materialien \\ 
\\
	   Absorption:\\
		-Wenn Materie/EM-Wellen in Materie aufgenommen werden\\
\\
	   Desorption:\\
		-Wenn Materie die Oberfläche eines Festkörpers verlässt, bzw aus der Flüssigen in die Gasphase übergeht\\
		-Umkehrprozess der Sorption\\
\\
	   Diffusion:\\
		-Der Prozess wenn ohne äußere Einwirkungen ein Konzentrationsunterschied sich ausgleicht.\\
		\\
	   "virtuelles" Leck:\\
	   	-Prozesse die das Vakuum reduzieren, jedoch von außen nicht messbar sind.\\
		- Ausgasung/Desorption/rückstände//
\\
	7. Methoden der Vakuumerzeugung:\\
		Funktionsweise von:\\
\\
		Drehschieberpumpe:\\
			-Das Volumen der Pumpkammer wird durch einen zylindrischen Rotor und 2 Drehschieber die durch Federn an die Wand gedrückt werden, das Volumen in drei Bereiche geteilt. //
			-Wenn der Rotor nun rotiert, wird gleichzeitig in einem Bereich neues Gas aus dem Rezipenten gezogen und in einem Anderem Bereich wird das Gas komprimiert und an einem Überdruckventil ausgegeben.//
			-p = o,5 * $10^{-1}$ mbar (Feinvakuum)
			-es liegt viskose laminare Stömung vor, der Innendurchmesser der Rohre kann also klein sein//
			//
		Turbomolekularpumpe://
			-mehstufige Turbine mit schaufelähnlichen Scheiben rotiert sehr schnell, ungefähr die mittlere thermische Geschwindigkeit der Teilchen.\\
			-die Teilchen werden beschleunigt und durch Abprallen an den Strator-Schaufeln durch die Pumpe geleitet.\\
			-Probleme bei leichten Gasen da die thermische Geschwindigkeit bereits sehr hoch ist\\

	   Methoden der Vakuummessung:\\
	   	Funktionsweise von:\\
\\
		Wärmeleitsungs-Vakuummeter:\\
			-Pirani-Vakuummetr:\\
				-arbeitet im Feimvakuum ($10^{-1} bis 10^{-3}mbar$)\\
				-nutz aus, dass Wärmeleitung im Bereich des Feinvakuums propotinal zum Druck ist\\
				-Wärmeleitung durch Stöße\\
				-Draht wird im Rezipenten mittels Strom aufgehitzt und die Temperatur des Drahtes gemessen indem der Widerstand gemessen wird.\\
				-Bei hohem druck kühlt der Draht schneller ab.\\
				\\
		Ionisations-Vakuummeter:\\
			Kaltkathode:\\
				-Penning-Vakuummeter:\\
					-Arbeitet im Hoch-und Ultrahochvakuum($10^{-3} bis 10^{-12}$mbar)\\
					-Glaskolben wird an Rezipienten angeschlossen und natürlich frei werdende Elektronen werden zwischen zwei Elektroden beschleunigt\\
					-Stomstärke ist Maß für Druck\\
					-Messgenauigkeit/Messpunkte werden durch ein eternes magnetfeld erhöht\\
					\\
			Glühkathode:\\
				-Bayard-Alpert-Vakuummeter:\\
					-Hoch und Ultrahochvakuum\\
					-Elektronenquelle ist eine Glühkathode\\
					-die beschleunigten Elektronen ionisieren Gasteilchen welche dann einen weiteren Strom erzeugen.\\
					-Stromstärke ist also Maß für das Vakuum.\\


	---> Auf Druckbereiche beachten!
	---> Vor-, und Nachteile beachten!


