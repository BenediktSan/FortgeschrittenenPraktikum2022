\newpage
\section{Diskussion}

\noindent
Bei der Durchführung des Versuchs sind ein paar Probleme aufgetreten. 
Die Bestimmung der optimalen Verzögerung lief reibungslos ab, für die Kalibrierung des MCA gab es Probleme mit der Schaltung, weswegen keine Messwerte mehr aufgenommen werden konnten.
Aus diesem Grund wurden alle weiteren Messwerte bereitgestellt und die Abschnitte ab dem Abschnitt \ref{seq:MCA} sind unabhängig von unseren Messungen.
Deswegen lässt sich der Verlauf der weiteren Messungen auch nicht diskutieren.\\
Die Lebensdauer der Myonen wurde über einen geeignet langen Zeitraum von über drei Tagen aufgenommen und liefert als Ergebnis die mittlere Lebensdauer $\tau = \SI{2.126(0019)}{\micro\second}$.
Der Literaturwert für die Lebensdauer ist $\tau_\t{theo} = \SI{2.1969811(00000022)}{\micro\second}$ \cite{PDG}.
Die relative Abweichung zwischen Literaturwert und errechnetem Wert lässt sich über die Formel
\begin{equation*}
    \\Delta x = \left| \frac{x - x_\t{theo}}{x_\t{theo}} \right|
\end{equation*}
bestimmen. Sie ergibt sich zu $\Delta \tau =\SI{-3.2(09)}{\percent} $. Daraus lässt sich erkennen, dass der errechnete Wert im Rahmen der Standardabweichung sehr gut mit dem Literaturwert übereinstimmt.
Dies lässt sich über die lange Messdauer und die damit im Verhältnis kleine Unsicherheit nach Poisson erklären. 
Außerdem tritt nur ein sehr kleiner Untergrund von Myonenzerfällen auf, welcher zu $U_\t{K} = 0.761$ bestimmt wurde.
Die trotzdem auftrettende Abweichung lässt sich unter anderem über die Einstellung der Verzögerungszeit vor einem der Diskriminatoren erklären. 
Dies wird während der Durchführung nur abgeschätzt und muss nicht den optimalen Wert ergeben. 
Des Weiteren ließe sich noch die Suchrate $T_S$ des Monoflops weiter feinjustieren um den Untergrund zu minimieren.\\ 
Insgesamt liefert der Versuch aber sehr gute Ergebnisse und die errechnete Lebensdauer stimmt sehr gut mit dem Literaturwert überein.
