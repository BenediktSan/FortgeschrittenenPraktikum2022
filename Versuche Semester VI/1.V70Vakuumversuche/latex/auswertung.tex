\newpage 
\section{Auswertung}

        \begin{table}[ht]
          \centering
          \small
          \caption{Messdaten zur Wien-Robinson-Brücke}
          \label{tab:tab2}
          \begin{tabular}{S [table-format=5.0] S [table-format=1.3]}
           \toprule
           {$\nu \mathbin{\scalebox{1.5} / } \si{\hertz}$} & $\text{Amplitude} \mathbin{\scalebox{1.5} / } \si{\volt}$\\
           \midrule
          \bottomrule
          \end{tabular}
        \end{table} 




        \subsection{Fehlerrechnung}
        \noindent
        Die Fortpflanzung von Messungenauigkeiten für mehrere unabhängige Fehler wird durch die Gaußsche Fehlerfortpflanzung
        \begin{equation*}
        \Delta f = \sqrt{\sum_{i \, = \, 1}^{n} \, \left(\frac{\partial f}{\partial x_i} \, \Delta x_i\right)^2}
        \label{fehler}
        \end{equation*}
        beschrieben. Dabei gibt $\Delta x$ die Unsicherheit des arithmetischen Mittelwerts $\bar{x}$ einer Observablen $x$ an:
        
        \begin{equation*}
        \Delta x =  \sqrt{\frac{1}{n(n-1)} \sum_{i \, = \, 1}^{n} \, \left(\bar{x}- x_i\right)^2}.
        \end{equation*}

        \noindent
        Die Zahl $n$ gibt die Anzahl der unabhängigen Messungen an.\\\\
        Die Messwerte, die bei Messungen mit der Turbopumpe aufgenommen wurden, besitzen im Bereich $\SI{1e-8}{\milli\bar}$ bis $\SI{100}{\milli\bar}$ eine Ungenauigkeit von $30$\%.
        Im Bereich von $\SI{100}{\milli\bar}$ bis $\SI{1000}{\milli\bar}$ sind es sogar $50$ \%.\\
        Für die Messungen mit der Drehschieberpumpe sind es für Werte kleiner als $\SI{2e-3}{\milli\bar}$ ein Faktor $2$ vom Messwert.
        Zusätzlich sind von $\SI{2e-3}{\milli\bar}$ bis $\SI{10}{\milli\bar}$ $\pm \SI{120}{\milli\bar}$ und von $\SI{10}{\milli\bar}$ bis $\SI{1200}{\milli\bar}$ $\pm \SI{3.6}{\milli\bar}$.\\\\
        Für die Fehlerfortpflanzung des logarhitmischen Fits wird die aus der Formel \ref*{fehler} hergeleitete Formel 
        \begin{equation*}
          \sigma = \sqrt{\frac{\sigma_p^2}{(p-p_E)^2}+\frac{\sigma_{p0}^2}{(p_0-p_E)^2}+\sigma_{p_E}^2 \left(\frac{1}{p_0-p_E}-\frac{1}{p-p_E}\right)^2}
        \end{equation*}
        genutzt.

        \subsection{Volumen}
        Um das Saugvermögen der beiden Pumpen zu bestimmen wird das Voluhmen des Versuchsaufbaus benötigt.\\
        Für die Messungen die für die Turbopumpe durchgeführt werden, wird ein kleineres Volumen benötigt, nämlich nur das bis hinter die Pumpe \ref{img:aufbaufoto}.
        Dieses beträgt $V_1 = \SI{33(033)}{\litre}$.\\
        Für die Messungen mit der Drehschieberpumpe wird zusätzlich noch das Volumen des Schlauchs genutzt. Dieses beträgt $\SI{1(01)}{\litre}$, wodurch sich das Volumen $V_2 = \SI{34(034)}{\litre}$ ergibt.\\



        \subsection{Drehschieberpumpe}

        \noindent Für die Drehschieberpumpe werden im folgenden beide Messverfahren ausgewertet. Einmal dass der Evakuierungsmessung und das der Leckratenmessung.\\
        Die Herstellerangabe des Saugvermögens der Drehschieberpumpe beträgt $ S_theo = \SI{1.1}{\litre\per\second}$.

        \subsubsection{Evakuierungsmessung}

        \noindent Für diese Messung wurden über einen Zeitraum von $ T = \SI{600}{\second}$ alle $ \increment t = \SI{10}{\second}$
        der Druck im Versuchsaufbau hinter der deaktivierten Turbomolekularpumpe abgelesen. Dies wurde drei mal wiederholt.\\
        Die Messdaten sind in der Tabelle \ref{tab:dreh_p} im Anhang, inklusive der gemittelten Messwerte mit Fehler des Mittelwerts, zu finden.\\\\
        
        \noindent Für die Auswertung wurde nun ein linearer Fit der Form
        \begin{equation*}
          y = m*x + n \quad
        \end{equation*}
        auf die logarhitmierten Messdaten, die nun die Form 
        \begin{equation*}
          \ln{\frac{p(t) - p_{end}}{p_0 - p_{end}}}
        \end{equation*}
        angewendet.\\
        Allerdings ist das Saugvermögen der Pumpe nicht über alle Druckbereiche konstant, was zum Beispiel an unterschiedlichen Ströumungsarten bei unterschiedlichen Drücken liegt.\\
        Aus diesem Grund wurde der Fit für drei unterschiedliche Bereiche durchgeführt. Die Bereiche wurden willkürlich grafisch abgelesen\\
        Das Ergebnis der Fits, die grafische Darstellung der Messdaten, inklusive Fehler, finden sich in Abbildung \ref{img:dreh_p}.
        Dabei sind auch die unterschiedlichen betrachteten Intervalle farblich hervorgehoben.\\
        \begin{figure}[h]
          \centering
          \includegraphics[width=0.7\textwidth]{build/plots/plot_dreh_p.pdf}
          \caption{Logarhitmische Darstellung der Evakuierungsmesswerte von der Drehschieberpumpe.\\
          Dabei sind die unterschiedlichen genutzten Intervalle farblich hervorgehoben und alle dazugehörigen Ausgleichsgeraden eingezeichnet.}
          \label{img:dreh_p}
        \end{figure}

        \noindent Der erste Bereich geht vom Startdruck bis $\SI{1.8}{\milli\bar}$. Der zweite bis $\SI{0.4387}{\milli\bar}$ und der dritte bis zum Ende.\\
        Aus den Fitparametern lässt sich über $ S = -mV $ das Saugvermögen bestimmen. Damit ergibt sich dann
        \begin{table}[H]
          \centering
          \small
          \caption{Parameter der Ausgleichsrechnungen und die Ergebnisse des Saugvermögens.}
          \label{tab:Saug_dreh_p}
          \begin{tabular}{S [table-format=5.0]  c c c}
           \toprule
           {} & $\text{m} \mathbin{\scalebox{1.5} / } \si{\per\second}$ & $\text{n}$ & $\text{S} \mathbin{\scalebox{1.5} / } \si{\litre\per\second}$ \\
           \midrule
           \text{Bereich} 1 & -0.0310 \pm 0.00024 & -0.1173 \pm 0.00024 & 1.0222 \pm 0.10252 \\
           \text{Bereich} 2 & -0.0110 \pm 0.00036 & -4.1260 \pm 0.0036 & 0.3617 \pm 0.03808\\
           \text{Bereich} 3 & -0.0055 \pm 0.00005 & -5.9464 \pm 0.00005 & 0.1828 \pm 0.01836 \\
          \bottomrule
          \end{tabular}
        \end{table} 



















      \newpage
        
        
        \begin{table}
          \begin{center}
            \begin{tabular}{cccccccc}
              \toprule
              $p_\text{Gg} \, / \, \text{mbar}$ & $a \, / \, \text{s}^{-1}$ &       $b$        & $Q \, / \, (\text{mbar \cdot l}/\text{s})$ & $S \, / \, (\text{l}/\text{s})$ &  \\ \midrule
              0,1                &    $0,0050 \pm 0,0002$    & $0,15 \pm 0,02$  &          $ 0,056 \pm 0,005  $           &       $ 0,56 \pm 0,07  $        &  \\
              0,4                &     $0,035 \pm 0,001$     & $0,33 \pm 0,05$  &          $  0,39 \pm 0,03   $           &       $  1,0 \pm 0,1   $        &  \\
              0,6                &     $0,067 \pm 0,002$     & $0,55 \pm 0,07 $ &           $ 0,74 \pm 0,06   $           &        $1,2 \pm 0,2   $         &  \\
              1,0                &    $0,1187 \pm 0,0007$    & $0,96 \pm 0,03 $ &           $  1,3 \pm 0,1   $            &       $  1,3 \pm 0,2   $        &  \\ \bottomrule
              &                           &
            \end{tabular}
            \caption{Die Parameter der Regressionsgeraden in Abbildung \ref{fig:drehLeck} für die Leckratenmessungen der Drehschieberpumpe.}
            \label{tab4}
          \end{center}
        \end{table}

        