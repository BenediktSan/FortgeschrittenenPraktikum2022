\newpage 
\section{Auswertung}

  \subsection{Spektrum der $^137\text{Cs}$-Quelle}
    Das Spektrum des Strahlers wurde in diesem Versuch mit einem NaI-Szintillationsdetektor aufgenommen. 
    In Abbildung \ref{fig:Spektrum} sind die Ergebnisse der Aufnahme in einem Histogramm dargestellt, auf der X-Achse sind die Channels des Multichannelanalyser abgebildet.
    \begin{figure}[H]
      \centering  
      \includegraphics[width = 0.4\linewidth]{python/build/spektrum.pdf}
      \label{fig:Spektrum}
      \caption{Das Aufgenommene Spektrum der Cs-Quelle.}
    \end{figure}

    \noindent In der Abbilldung \ref{fig:Spektrum} sind nur die Anzahl der Ereignisse gegen die Channel aufgetragen. 
    Die Energie der Strahlung wird von dem Multichannelanalyser nicht vermessen. 
    Jedoch lässt sich aus der Literatur \cite{Peak} entnehmen, dass der Photopeak des $^173\t{Cs}$-Strahlers bei $\SI{662}{\kilo\electronvolt}$ zu finden ist.
    Dieser Photpeak der Strahlungsquelle ist in Abbildunf \ref{fig:Spektrum} ca. bei Channel 65 zu finden. Bei Channel 43 ist die Comptonkante und bei Channel 21 ist der Rückstrahlpeak.

  \subsection{Würfel 1}
    Zunächst wird die Nullrate bestimmt welche durch die Aluminiumhülle dringt und alle folgenden Würfel umgibt. 
    Die in dieser Messreihe bestimmten Intensitäten des Photopeaks wird daher als Nullrate $N_0$ für die folgenden Messungen benutzt.
    Hier wird dazwischen unterschieden wie der Strahl durch den Block dringt, entweder gerade, also senkrecht zu einer Zeite, oder auf einer Haupt- oder Nebendiagonalen.
    Da es sich hier um ein Zählexperiment handelt, ist der Fehler Poissoncerteilt: $\sigma_N = \sqrt{N}$.
    Auswertung der Netarea mit den Messdaten aus \ref{tab:M_B1} ergibt für die Nullrate die Werte aus Tabelle \ref{tab:w1}.
    \begin{table}[H]
    \centering
     \caption{Die gemessene Anzahl der Ereignisse und die entsprechende Zählrate der Messung des leeren Würfel 1, der nur aus der Aluminiumhülle besteht.}
     \label{tab:w1}
     \begin{tabular}{c S[table-format=4.0] @{${}\pm{}$} S[table-format=2.1]} 
         \toprule
         {Strahlengang} & \multicolumn{2}{c}{$N$}  \\
         \midrule
         Gerade                 &  3704 & 190  \\ 
         Hauptdiagonale         &  3636 & 190  \\ 
         Nebendiagonale         &  3796 & 190  \\ 
         \bottomrule 
     \end{tabular}
    \end{table}
  
  \subsection{Würfel 2}
    Da Würfel 2 homogen ist, und aufgrund der geringen Dichte große Statistik erzeugt, wurden für diesen Block nur 2 Messreihen durchgeführt.
    Mit einer Netarea von $N_G=30323$ für die Gerade und$N_D=29521$ für die Diagonale berechnen sich die Absorptionskoeffizienten zu $\mu_G = \SI{0.0699(00026)}{1\per\centi\metre}$ und $\mu_D=\SI{0.0491(00018)}{1\per\centi\metre}$.
    Dazu wird die Formel
    \begin{equation}
      \mu = \frac{\t{ln}\left(\frac{N_0}{N}\right)}{d} \, \t,
    \end{equation}
    mit der Nullrate $N_0$ der Netarea $N$ und der Länge $d$ welche die Strahlung innerhalb des Alluminiumgehäuses durchquert.
    Im Mittel ergibt sich für Block 2 ein Absorptionskoeffizienten von: 
    \begin{equation}
      \mu_\t{Würfel 2} = \SI{0.0595(00016)}{1\per\centi\metre}.
    \end{equation}

  \subsection{Würfel 3}
    Würfel 3 ist auch homogen, jedoch ließ seine hohe Dichte bereits auf einen hohen Absorptionskoeffizienten schließen. 
    Um bessere Werte sicherzustellen, wurden daher für diesen Block mehr MEssreihen Durchgeführt:
    \begin{table}[H]
     \centering
     \caption{Die Messwerte und daraus errechneten Werte der Messung des Würfel 3.}
     \label{tab:w2}
     \begin{tabular}{c S[table-format=4.0] S[table-format=1.3] @{${}\pm{}$} S[table-format=1.3]}
       \toprule
       {Projektion} &\multicolumn{1}{c}{$N$}  & \multicolumn{2}{c}{$\mu \mathbin{/} \left(\si{\per\centi\metre}\right)$} \\
       \midrule
       Gerade & 1330 & 1.112 & 0.009 \\
       Gerade & 1375 & 1.101 & 0.009 \\
       Gerade & 1366 & 1.103 & 0.009 \\
       Diagonale & 614 & 0.962 & 0.010 \\
       Diagonale & 568 & 0.980 & 0.010 \\
       Nebendiagonale & 2258 & 0.998 & 0.008 \\
       \bottomrule  
     \end{tabular}
    \end{table} 
    Im Mittel ergibt dies einen Absorptionskoeffizienten von $\mu_\t{Würfel 3} = \SI{1.045(0004)}{1\per\centi\metre}$.

  \subsection{Würfel 4}
    Würfel 4 besteht insgesamt aus 27 kleineren jeweils homogenen Würfeln.
    In diesem Versuch wird eine horizontale Ebene mit 9 Würfeln mittels 12 Projektionen ausgewertet.
    Die Projektionen und die Zugehörige Matrix $A$ sind in der Durchführung erklärt und dargestellt.
    Mittels dem Vektor an Intensitäten $\vec{I}$ ergeben sich nach der Formel des Kleinste-Quadrate Fittes:
    \begin{equation}
      \vec{\mu} = \left( A^T \cdot A \right)^-1 \cdot A^T \vec{I}
    \end{equation}
    die einezelnen Absorptionskoeffizienten für die 9 kleinen Blöcke.
    In dieser Transformation, transformieren sich die Unsicherheiten nach der Formel:
    \begin{equation}
      V[\vec{\mu}] = \left( A^T \cdot A \right)^-1 \cdot A^T V[\vec{I}]A \cdot \left( \left( A^T \cdot A \right)^-1 \right)^T \,.
    \end{equation}
    $V[\vec{\mu}]$ ist hier eine Kovarianzmatrix welche die Unsicherheiten der einzelnen Absorptionskoeffizienten $\mu_i$ auf der Hauptdiagonalen hat.
    In Tabelle  \ref{tab:w4} sind nun die Projektionen mit den gemessenen Events und den nach
    \begin{equation}
      I_i = \t{ln}\left( \frac{N_{G/D/N}{N_i}} \right)
    \end{equation}
    berechneten Intensitäten. $N_{G/D/N}$ steht hier für die Nullrate, je nachdem ob der Strahl gerade($G$) durch den Block geht oder eine Haupt($D$)-, oder Nebendiagonale($N$) durchstrahlt.
    \begin{table}[H]
    \centering
    \caption{Die gemessenen Anzahlen der Ereignisse unter dem Photopeak und die daraus errechneten Werte $I_i$ von der Messung des Würfel 4.}
    \label{tab:w4}
    \begin{tabular}{c S[table-format=4.0] S[table-format=2.3] @{${}\pm{}$} S[table-format=1.3] }
      \toprule
      {Projektion} & \multicolumn{1}{c}{$N$} & \multicolumn{2}{c}{$I \mathbin{/} \left(\si{\per\second}\right)$} \\
      \midrule
      $I_1 $   &  10972 & 1.226 & 0.010 \\
      $I_2 $   &  11123 & 1.212 & 0.010\\
      $I_3 $   &  10915 & 1.231 & 0.010\\
      $I_4 $   &  29942 & 0.222 & 0.007\\
      $I_5 $   &   1475 & 3.233 & 0.026\\
      $I_6 $   &  29503 & 11.968& 0.029 \\
      $I_7 $   &   6745 & 1.712 & 0.013\\
      $I_8 $   &  58814 & 1.861 & 0.014\\
      $I_9 $   &   7040 & 1.670 & 0.012\\
      $I_{10}$ &   7959 & 1.547 & 0.012\\
      $I_{11}$ &   7281 & 1.636 & 0.012\\
      $I_{12}$ &   9338 & 1.387 & 0.011\\
      \bottomrule
    \end{tabular}
  \end{table}
  Die mit diesen Werten berechneten Absorptionskoeffizienten sind in Tabelle \ref{tab:w5} zu finden.
  Da diese Ergebnisse in Teilen unphysikalisch sind, werden erst in der Diskussion physikalische Interpretationen der Ergebnisse durchgeführt.

  \begin{table}[H]
    \centering
    \caption{Die ermittelten Werte für die Absorptionskoeffizienten der verschiedenen kleineren Würfel.}
    \label{tab:mu4}
    \begin{tabular}{c S[table-format=1.4] @{${}\pm{}$} S[table-format=1.6] }
      \toprule
      {Würfel} & \multicolumn{2}{c}{$\mu \mathbin{/} \left(\si{\per\centi\metre}\right)$}  \\
      \midrule
      1 & -1.8506& 0.024145 & 0.211 & Aluminium \\
      2 & 0.3600 & 0.023789 & 1.415 & Blei \\
      3 & 4.0970 & 0.024831 & 0.211 & Aluminium \\
      4 & 1.2653 & 0.020428 & 0.121 & Delrin \\
      5 & -0.4721& 0.025784 & 1.415 & Blei \\
      6 & 1.2207 & 0.020524 & 0.121 & Delrin \\
      7 & -1.7298& 0.023762 & 0.121 & Delrin \\
      8 & 0.2287 & 0.023457 & 1.415 & Blei \\
      9 & 4.1127 & 0.023045 & 0.121 & Delrin \\
      \bottomrule
    \end{tabular}
  \end{table}1.2263776058511713+\-0.010857146243594119