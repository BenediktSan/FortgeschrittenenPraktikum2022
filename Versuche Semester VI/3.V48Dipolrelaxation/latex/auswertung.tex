\newpage 
\section{Auswertung}

\noindent Es wurden im Abstand von einer Minute der Depolarisationsstrom und die Temperatur der Probe gemessen.
Die Heizspule wurde dabei so eingestellt, dass sich für die beiden Messreihen unterschiedliche Heizraten $b$ ergeben.
Die Heizraten lassen sich über das Mitteln der Differenzen zwischen den Temperaturen bestimmen.\\
\begin{equation*}
  b = \frac{1}{n-1} \sum_{i=1}^{n} \frac{T_i - T_{i-1}}{\SI{1}{\minute}}
\end{equation*}
Zum Berechnen der Heizraten für die einzelnen Messreihen werden die Werte aus der Tabelle \ref{tab:mess1} und der Tabelle \ref{tab:mess2} genutzt.\\
Mit dem Fehler des Mittelwerts als Abweichung ergibt sich dann für die Messreihe mit $\increment T = \SI{1.5}{\kelvin}$ für die Heizrate
\begin{equation*}
  b_1 = \SI{1.36(016)}{\kelvin\per\minute}
\end{equation*}
und für die Messreihe mit $\increment T = \SI{2}{\kelvin}$ 
\begin{equation*}
  b_2 = \SI{1.94(025)}{\kelvin\per\minute} \quad .
\end{equation*}

\subsection{Heizrate $\textbf{b = \SI{1.5}{\kelvin\per\minute}}$}
\label{sec:15}

\noindent
Um aus dem Depolarisationsstrom die Relaxationszeit zu bestimmen, muss zuerst der Untergrund von den gemessenen Strömen entfernt werden.
Dafür wird auf grafisch abgeschätzte Messwerte eine Exponentialfunktion der Form
\begin{equation}
  I_\t{Unter}(T) = A\cdot \exp\left(\frac{-b}{T}\right)
  \label{eqn:exp}
\end{equation}
gefittet. Anschließend wird der Untergrund von den Messwerten abgezogen, um sie zu bereinigen
\begin{equation*}
  \tilde{I}(T_i) = I(T_i) - I_\t{Unter}(T_i) \quad.
\end{equation*}
Für den Fit wurden die Messwerte aus den Zeitintervallen $t \in [0,4] \; \si{\minute}$ und $t \in [28,61] \; \si{\minute}$ genutzt.
Als Fitparameter ergeben sich so 
\begin{align*}
  A &= \SI{0.15(098)}{\ampere}\\
  b &= \SI{6200.71(19659)}{\kelvin} \quad .
\end{align*}
Die zur Bestimmung des Untergrunds genutzten Messwerte sind,
inklusive der gesamten Messwerte und des Fits, in Abbildung \ref{img:mitunter15} grafisch dargestellt. 
Die genutzten Messwerte inklusive der vom Untergrund bereinigten Werte für den Strom sind in Tabelle \ref{tab:mess1} zu finden.
\begin{figure}[h]
  \centering
  \includegraphics[width=0.7\textwidth]{build/plots/mitunter_1.5grad.pdf}
  \caption{Die Messwerte für eine Heizrate von $b_1 = \SI{1.36}{\kelvin\per\minute}$. 
  Außerdem sind die Werte, die für den Untergrundfit genutzt wurden, und der Untergrundfit eingezeichnet.}
  \label{img:mitunter15}
\end{figure}

\noindent
Der Depolarisationsstrom ohne Untergrund ist in Abbildung \ref{img:ohneunter15} grafisch dargestellt. 
Zusätzlich sind noch die Bereiche, die für die Auswertung über den Polarisations und über den Integrationsansatz hervorgehoben.
\begin{figure}[H]
  \centering
  \includegraphics[width=0.7\textwidth]{build/plots/ohneunter_1.5grad.pdf}
  \caption{Die bereinigten Messwerte für eine Heizrate von $b_1 = \SI{1.36}{\kelvin\per\minute}$ 
          mit den Bereichen, die für den Polarisations und den Integrationansatz genutzt werden, hervorgehoben.}
  \label{img:ohneunter15}
\end{figure}

%{\centering$\text{Depolarisations-}$\\$\text{strom } $\\$ \text{in }\si{\pico\ampere}$}
\begin{table}[H]
  \small
  \centering
  \begin{tabular}{S [table-format=3.0] S [table-format=3.2] S [table-format=2.3] c |S [table-format=3.0] S [table-format=3.2] S [table-format=2.3] c }
      \toprule
      \multicolumn{1}{p{1.5cm}}{\centering$\text{t} $\\$ \text{in } \si{\minute} $ } &
      \multicolumn{1}{p{1.5cm}}{\centering$\text{Temperatur } $\\$ \text{in }\si{\kelvin}$} &
      \multicolumn{1}{p{1.5cm}}{\centering$I $\\$ \text{in }\si{\pico\ampere}$} &
      \multicolumn{1}{p{1.5cm}}{\centering$I_{\t{bereinigt}} $\\$ \text{in }\si{\pico\ampere}$} \vline&
      \multicolumn{1}{p{1.5cm}}{\centering$\text{t} $\\$ \text{in } \si{\minute} $ } &
      \multicolumn{1}{p{1.5cm}}{\centering$\text{Temperatur } $\\$ \text{in }\si{\kelvin}$} &
      \multicolumn{1}{p{1.5cm}}{\centering$I $\\$  \text{in }\si{\pico\ampere}$} &
      \multicolumn{1}{p{1.5cm}}{\centering$\tilde{I}_{\t{bereinigt}} $\\$ \text{in }\si{\pico\ampere}$} \\
      \midrule
       0   & 229.15 &  0.165 &  0.162   & 35   & 275.45 &  0.275 &  0.020  \\  
       1   & 230.55 &  0.210 &  0.206   & 36   & 276.85 &  0.31  &  0.025  \\  
       2   & 231.55 &  0.280 &  0.276   & 37   & 278.15 &  0.33  &  0.013  \\  
       3   & 233.05 &  0.380 &  0.375   & 38   & 279.45 &  0.35  & -0.0004 \\  
       4   & 234.45 &  0.470 &  0.465   & 39   & 280.75 &  0.36  & -0.028  \\  
       5   & 235.75 &  0.590 &  0.584   & 40   & 282.15 &  0.42  & -0.013  \\  
       6   & 237.15 &  0.710 &  0.703   & 41   & 283.45 &  0.45  & -0.029  \\  
       7   & 238.55 &  0.900 &  0.892   & 42   & 284.65 &  0.49  & -0.035  \\  
       8   & 239.95 &  1.150 &  1.140   & 43   & 286.05 &  0.54  & -0.045  \\  
       9   & 241.15 &  1.450 &  1.439   & 44   & 287.35 &  0.60  & -0.045  \\  
      10   & 242.45 &  1.850 &  1.838   & 45   & 288.75 &  0.66  & -0.056  \\  
      11   & 243.75 &  2.400 &  2.386   & 46   & 290.15 &  0.74  & -0.054  \\  
      12   & 245.05 &  3.200 &  3.184   & 47   & 291.55 &  0.81  & -0.070  \\  
      13   & 246.35 &  4.100 &  4.082   & 48   & 292.95 &  0.91  & -0.064  \\  
      14   & 247.55 &  5.400 &  5.379   & 49   & 294.45 &  1.05  & -0.036  \\  
      15   & 248.75 &  7.000 &  6.977   & 50   & 295.85 &  1.15  & -0.049  \\  
      16   & 249.95 &  8.700 &  8.674   & 51   & 297.35 &  1.30  & -0.033  \\  
      17   & 251.25 & 10.500 & 10.470   & 52   & 298.85 &  1.45  & -0.030  \\  
      18   & 252.45 & 12.500 & 12.467   & 53   & 300.25 &  1.65  &  0.018  \\  
      19   & 253.55 & 14.000 & 13.963   & 54   & 301.65 &  1.85  &  0.054  \\  
      20   & 254.75 & 15.000 & 14.959   & 55   & 303.05 &  2.05  &  0.075  \\  
      21   & 255.95 & 12.500 & 12.454   & 56   & 304.35 &  2.25  &  0.095  \\  
      22   & 257.25 &  6.500 &  6.448   & 57   & 305.85 &  2.45  &  0.069  \\  
      23   & 258.55 &  5.400 &  5.341   & 58   & 307.15 &  2.65  &  0.055  \\  
      24   & 259.75 &  4.000 &  3.934   & 59   & 308.55 &  2.85  &  0.007  \\  
      25   & 261.05 &  3.200 &  3.126   & 60   & 310.05 &  3.00  & -0.133  \\  
      26   & 262.55 &  1.450 &  1.365   & 61   & 311.45 &  3.10  & -0.327  \\  
      27   & 263.95 &  0.450 &  0.354   & 62.5 & 313.65 &  3.20  & -0.741  \\  
      28   & 265.35 &  0.305 &  0.197   & 63   & 314.45 &  3.20  & -0.944  \\  
      29   & 266.75 &  0.260 &  0.138   & 64   & 315.75 &  3.10  & -1.395  \\  
      30   & 268.25 &  0.240 &  0.101   & 65   & 317.25 &  2.95  & -1.982  \\  
      31   & 269.75 &  0.230 &  0.072   & 66   & 318.75 &  2.75  & -2.658  \\  
      32   & 271.25 &  0.230 &  0.050   & 67   & 320.25 &  2.55  & -3.374  \\  
      33   & 272.65 &  0.240 &  0.038   & 68   & 321.75 &  2.25  & -4.233  \\  
      34   & 273.95 &  0.255 &  0.030   & 69   & 323.25 &  1.95  & -5.140  \\  
      \bottomrule 
      \end{tabular}
      \caption{Messwerte der Depolarisationsstrommessung und für die vom Untergrund bereinigten Depolarisationsströme, bei einer Heizrate von $b = \SI{1.36}{\kelvin}$. }
      \label{tab:mess1}
\end{table}



\subsubsection{Ausgleichsrechnung über den Polarisationsansatz}

\noindent
Um die Aktivierungsenergie $W$ zu bestimmen wird eine lineare Ausgleichsrechnung auf dem Intervall des Depolarisationsstroms, vom Ende der genutzten Untergrundwerte bis zum Maximalwert der Termperatur, berechnet.
Das Maximum liegt dabei bei $T(t_\t{max} = 20) = \SI{254.75}{\kelvin}$. 
Zum Fitten werden die logarithmisch aufgetragenen Depolarisationsströme gegen das reziproke der Temperatur auf dem Intervall $t \in [5,20]$, mit einer linearen Funktion der Form
\begin{equation}
  y(T) = m\cdot \frac{1}{T} + n \quad,
  \label{eqn:lin}
\end{equation}
\noindent
genutzt. Für die Parameter ergibt sich dabei
\begin{align*}
  m & = \SI{-10956.62 (19462)}{\kelvin}\\
  n & = \SI{20.55(080)}{} \quad .
\end{align*}
Die Messwerte und die damit korrespondierende Ausgleichsgerade sind in Abbildung \ref{img:pol15} grafisch dargestellt.\\
Nach Gleichung \ref{eqn:W} lässt sich über $W = -m \cdot k_\t{B}$, mit $k_\t{B}$\cite{kb} als Boltzmannkonstante, die Aktivierungsenergie zu 
\begin{equation*}
  W = \SI{0.944(0017)}{\electronvolt}
\end{equation*}
berechnen. Aus der Gleichung \ref{eqn:taumaxsource} lässt sich der Zusammenhang 
\begin{align}
  \tau_\t{max} &= \frac{k_\t{B}\cdot T_\t{max}^2}{b W}
  \label{eqn:taumax}\\
  \tau_\t{max} &= \SI{4.34 (053)}{\second} \notag
\end{align}
herleiten, wobei $b$ die Heizrate ist und $T_\t{max}$ der Temperaturwert, der mit dem maximalen Strom korrespondiert. 
Daraus lässt sich die maximale Relaxationszeit bestimmen.
Damit wird in Abschnitt \ref{sec:tau} die charakteristische Relaxationszeit $\tau_0$ bestimmt.

\begin{figure}[ht]
  \centering
  \includegraphics[width=0.7\textwidth]{build/plots/asc_1.5grad.pdf}
  \caption{Die Messwerte für eine Heizrate von $b_1 = \SI{1.36}{\kelvin\per\minute}$ 
          logarithmisch gegen $\frac{1}{T}$ aufgetragen.}
  \label{img:pol15}
\end{figure}



\subsubsection{Ausgleichsrechnung über den Integrationsansatz}
\label{sec:int}

\noindent 
Um über den Integrationsansatz die Aktivierungsenergie zu bestimmen wird im Vergleich mit Gleichung \ref{eqn:int} ein linearer Fit in 
$\frac{1}{T}$ gegen die nach Gleichung \ref{eqn:int} integrierten Werte genutzt. Dabei wurden die bereinigten Depolarisationsstromwerte
des Intervalls $t \in [4,18]$  mit der Trapezregel numerisch von ihrem korrespondierenden $T$-Wert bis $\frac{1}{T(t = 4)}$ aufintegriert.\\
Der Fit und die integrierten Werte sind in Abbildung \ref{img:int15} aufgetragen.\\
Als Fitparameter ergeben sich 
\begin{align*}
  m & = \SI{ -8480.55(99864)}{\kelvin}\\
  n & = \SI{36.10 (401)}{} \quad .
\end{align*}
Über den Vergleich mit Gleichung \ref{eqn:int} lässt sich die Energie, über $W = -m k_\t{B}$, zu 
\begin{equation*}
  W = \SI{0.731(0086)}{\electronvolt}
\end{equation*}
bestimmen. Damit ergibt sich nach Gleichung \ref{eqn:taumax} für die maximale Relaxationszeit 
\begin{equation*}
  \tau_\t{max} = \SI{5.61 (095)}{\second} \quad .
\end{equation*}

%\subsubsection{Bestimmung von $W$ und $\tau _{max}$}
\begin{figure}[ht]
  \centering
  \includegraphics[width=0.7\textwidth]{build/plots/asc_1.5grad.pdf}
  \caption{Die integrierten Messwerte für eine Heizrate von $b_1 = \SI{1.36}{\kelvin\per\minute}$ 
          logarithmisch gegen $\frac{1}{T}$ aufgetragen.}
  \label{img:int15}
\end{figure}

\subsection{Heizrate $\textbf{b = \SI{2}{\kelvin\per\minute}}$}

\noindent
Die Auswertung für die Messreihe mit $b_2 = \SI{1.94(025)}{\kelvin\per\minute}$ ist analog zu der in Abschnitt \ref{sec:15}. 
Auf die Messwerte des Depolarisationsstroms, die in Tabelle \ref{tab:mess2} zu finden sind, wird nach Gleichung \ref{eqn:exp} ein Fit auf den Untergrund berechnet.\\
Dafür werden die Messwerte aus den Zeitintervallen $t \in [0,13]\si{\minute}$ und $t \in [33,56]\si{\minute}$ gefittet.
Die Parameter ergeben sich zu 
\begin{align*}
  A &= \SI{ 0.12  (005)}{\ampere}\\
  b &= \SI{ 6110.98 (13422)}{\kelvin} \quad .
\end{align*}
Der Untergrundfit inklusive der Messwerte aus Tabelle \ref{tab:mess2} sind in Abbildung \ref{img:mitunter2} dargestellt.
Die bereinigten Messwerte sind in Tabelle \ref{tab:mess2} zu finden, sowie auch grafisch in Abbildung \ref{img:ohneunter2}.
In der Abbildung sind ebenfalls wieder die Intervalle für die weiteren Auswertungschritte hervorgehoben.
\begin{figure}[h]
  \centering
  \includegraphics[width=0.7\textwidth]{build/plots/mitunter_2grad.pdf}
  \caption{Die Messwerte für eine Heizrate von $b_2 = \SI{1.94}{\kelvin\per\minute}$. 
  Außerdem sind die Werte, die für den Untergrundfit genutzt wurden, und der Untergrundfit eingezeichnet.}
  \label{img:mitunter2}
\end{figure}

\begin{figure}[h]
  \centering
  \includegraphics[width=0.7\textwidth]{build/plots/ohneunter_2grad.pdf}
  \caption{Die bereinigten Messwerte für eine Heizrate von $b_2 = \SI{1.94}{\kelvin\per\minute}$ 
  mit den Bereichen, die für den Polarisations und den Integrationansatz genutzt werden, hervorgehoben.}
  \label{img:ohneunter2}
\end{figure}

\begin{table}[H]
  \small
  \centering
  \begin{tabular}{S [table-format=3.0] S [table-format=3.2] S [table-format=2.3] c |S [table-format=3.0] S [table-format=3.2] S [table-format=2.3] c }
      \toprule
      \multicolumn{1}{p{1.5cm}}{\centering$\text{t} $\\$ \text{in } \si{\minute} $ } &
      \multicolumn{1}{p{1.5cm}}{\centering$\text{Temperatur } $\\$ \text{in }\si{\kelvin}$} &
      \multicolumn{1}{p{1.5cm}}{\centering$I $\\$ \text{in }\si{\pico\ampere}$} &
      \multicolumn{1}{p{1.5cm}}{\centering$I_{\t{bereinigt}} $\\$ \text{in }\si{\pico\ampere}$} \vline&
      \multicolumn{1}{p{1.5cm}}{\centering$\text{t} $\\$ \text{in } \si{\minute} $ } &
      \multicolumn{1}{p{1.5cm}}{\centering$\text{Temperatur } $\\$ \text{in }\si{\kelvin}$} &
      \multicolumn{1}{p{1.5cm}}{\centering$I $\\$  \text{in }\si{\pico\ampere}$} &
      \multicolumn{1}{p{1.5cm}}{\centering$\tilde{I}_{\t{bereinigt}} $\\$ \text{in }\si{\pico\ampere}$} \\
      \midrule
       0 & 205.35 &  0.015 &  0.015    &  32 & 267.65 &  0.510 &  0.359 \\ 
       1 & 206.35 &  0.015 &  0.015    &  33 & 269.75 &  0.400 &  0.219 \\ 
       2 & 207.95 & -0.007 & -0.007    &  34 & 271.75 &  0.350 &  0.136 \\ 
       3 & 209.95 & -0.010 & -0.010    &  35 & 273.75 &  0.320 &  0.068 \\ 
       4 & 211.95 & -0.080 & -0.080    &  36 & 275.65 &  0.350 &  0.057 \\ 
       5 & 214.05 &  0.015 &  0.015    &  37 & 277.55 &  0.380 &  0.039 \\ 
       6 & 216.15 &  0.025 &  0.024    &  38 & 279.65 &  0.420 &  0.017 \\ 
       7 & 218.35 &  0.030 &  0.029    &  39 & 281.25 &  0.460 &  0.004 \\ 
       8 & 220.55 &  0.050 &  0.049    &  40 & 283.25 &  0.520 & -0.010 \\ 
       9 & 222.25 &  0.060 &  0.059    &  41 & 285.15 &  0.590 & -0.022 \\ 
      10 & 224.15 &  0.090 &  0.088    &  42 & 287.15 &  0.680 & -0.031 \\ 
      11 & 225.95 &  0.125 &  0.123    &  43 & 289.15 &  0.780 & -0.044 \\ 
      12 & 228.05 &  0.160 &  0.157    &  44 & 291.05 &  0.880 & -0.066 \\ 
      13 & 230.15 &  0.220 &  0.216    &  45 & 293.05 &  1.000 & -0.092 \\ 
      14 & 232.65 &  0.290 &  0.285    &  46 & 294.85 &  1.150 & -0.090 \\ 
      15 & 234.85 &  0.350 &  0.344    &  47 & 296.75 &  1.350 & -0.066 \\ 
      16 & 237.05 &  0.450 &  0.442    &  48 & 298.55 &  1.550 & -0.053 \\ 
      17 & 239.25 &  0.600 &  0.590    &  49 & 300.45 &  1.800 & -0.025 \\ 
      18 & 241.25 &  0.800 &  0.788    &  50 & 302.65 &  2.100 & -0.016 \\ 
      19 & 243.15 &  1.000 &  0.985    &  51 & 304.85 &  2.500 &  0.051 \\ 
      20 & 244.95 &  1.400 &  1.381    &  52 & 306.85 &  2.950 &  0.159 \\ 
      21 & 246.65 &  1.800 &  1.778    &  53 & 309.05 &  3.400 &  0.185 \\ 
      22 & 248.35 &  2.400 &  2.374    &  54 & 310.85 &  3.600 & -0.005 \\ 
      23 & 250.15 &  3.000 &  2.969    &  55 & 312.75 &  3.900 & -0.162 \\ 
      24 & 252.15 &  3.400 &  3.363    &  56 & 314.75 &  4.000 & -0.599 \\ 
      25 & 254.15 &  4.500 &  4.456    &  57 & 316.55 &  4.000 & -1.136 \\ 
      26 & 256.15 &  5.200 &  5.146    &  58 & 318.35 &  3.900 & -1.828 \\ 
      27 & 257.95 &  5.700 &  5.636    &  59 & 320.15 &  3.500 & -2.881 \\ 
      28 & 259.85 &  3.000 &  2.924    &  60 & 321.85 &  3.100 & -3.958 \\ 
      29 & 261.65 &  3.200 &  3.111    &  61 & 323.65 &  2.600 & -5.244 \\ 
      30 & 263.45 &  3.200 &  3.095    &  62 & 325.35 &  2.200 & -6.457 \\ 
      31 & 265.55 &  1.000 &  0.874    &     &        &        &        \\
      \bottomrule 
      \end{tabular}
      \caption{Messwerte der Depolarisationsstrommessung und für die vom Untergrund bereinigten Depolarisationsströme, bei einer Heizrate von $b = \SI{1.94}{\kelvin}$. }
      \label{tab:mess2}
\end{table}


\subsubsection{Ausgleichsrechnung über den Polarisationsansatz}

\noindent
Für den Polarisationsansatz wird analog wieder eine lineare Ausgleichsrechnung durchgeführt.
Dafür werden Werte aus dem Intervall $t \in [13,27]$ genutzt.
Das Maximum der Messreihe liegt in diesem Intervall und bei $T(t_\t{max} = 27) = \SI{257.95}{\kelvin}$. \\
Der Fit der Messwerte nach Gleichung \ref{eqn:lin}, auf dem logarithmierten Strom gegen $\frac{1}{T}$ aufgetragen, führt zu den Parametern
\begin{align*}
  m & = \SI{ -5602.33( 74537)}{\kelvin}\\
  n & = \SI{24.71(300)}{} \quad .
\end{align*}
Das Ergebnis der Ausgleichsrechnung ist zusammen mit den beschriebenen Werten in Abbildung \ref{img:pol2} grafisch dargestellt.

\begin{figure}[ht]
  \centering
  \includegraphics[width=0.7\textwidth]{build/plots/asc_2grad.pdf}
  \caption{Die Messwerte für eine Heizrate von $b_1 = \SI{1.94}{\kelvin\per\minute}$ 
          logarithmisch gegen $\frac{1}{T}$ aufgetragen.}
  \label{img:pol2}
\end{figure}

\noindent
Aus den Parametern des Fits lässt sich über die Gleichungen $W = -k_\t{B} m$ und Gleichung $\ref{eqn:taumax}$ die Aktivierungsenergie und die maximale Relaxationszeit bestimmen.
Dies führt zu 
\begin{align*}
  W &= \SI{0.65 (002)}{\electronvolt}\\
  \tau_\t{max} &= \SI{4.58 (059)}{\second} \quad .
\end{align*}



\subsubsection{Ausgleichsrechnung über den Integrationsansatz}

\noindent
Die Aktivierungsenergie bestimmt sich hier über das aufintegrieren der bereinigten Messwerte für den Strom. 
Dies geschieht analog zu der in Abschnitt \ref{sec:int} durchgeführten Rechnung.
Dabei werden die Messwerte aus dem Intervall  $t \in [13,33]$ verwendet.
Als Parameter für die lineare Funktion ergeben sich damit die Werte
\begin{align*}
  m & = \SI{  -5602.33 (74537)}{\kelvin}\\
  n & = \SI{24.71 (300)}{} \quad .
\end{align*}
Daraus lassen sich dann die Werte 
\begin{align*}
  W &= \SI{ 0.49  (006)}{\electronvolt}\\
  \tau_\t{max} &= \SI{6.14 (113)}{\second}
\end{align*}
bestimmen. Die integrierten Stromwerte sind in Abbildung \ref{img:int2}, inklusive des dazugehörigen Fits, gegen $\frac{1}{T}$ aufgetragen.

\begin{figure}[ht]
  \centering
  \includegraphics[width=0.7\textwidth]{build/plots/int_2grad.pdf}
  \caption{Die integrierten Messwerte für eine Heizrate von $b_1 = \SI{1.94}{\kelvin\per\minute}$ 
          logarithmisch gegen $\frac{1}{T}$ aufgetragen.}
  \label{img:int2}
\end{figure}


\subsection{Bestimmung von \textbf{\texorpdfstring{$\tau_0$}{tau_0}}}
\label{sec:tau}

\noindent
Um die charakteristische Relaxationszeit $\tau_0$ zu bestimmen, wird die Gleichung \ref{eqn:tau0} genutzt und nach $\tau_0$ umgeformt.
Dies führt für die Temperatur $T_\t{max}$ zu 
\begin{equation*}
 \tau_0 =   \tau(T_\t{max})\text{exp}\left(\frac{W}{k_\text{B}T_\t{max}}\right) \quad .
\end{equation*}
Die maximalen Temperaturen sind dabei für die erste Messreihe $\SI{254.75}{\kelvin}$ und $\SI{257.95}{\kelvin}$ für die zweite.
Die einzusetzenden Werte und die dazugehörigen Ergebnisse sind in Tabelle \ref{tab:tau} abgebildet.
\begin{table}[H]
  \centering
  \small
  \begin{tabular}{S [table-format=9.0]  c c l}
   \toprule
   {Auswertungsverfahren} & $\text{W} \mathbin{\scalebox{1.5} / } \si{\electronvolt}$ & $\tau_{max} \mathbin{\scalebox{1.5} / } \si{\second}$  & $\text{    }\tau_0 \mathbin{\scalebox{1.5} / } \si{\atto\second}$ \\
   \midrule
   \text{Polarisation $\SI{1.36}{\kelvin}$ }  & 0.9442 \pm 0.0168  & 4.3432 \pm 0.5285  & $\SI{0.91 (072)}{}$  \\
   \text{Polarisation $\SI{1.94}{\kelvin}$ }  & 0.7308 \pm 0.0861  & 5.6113 \pm 0.9450  & $\SI{19.57 (7905)e3}{}$ \\
   \text{Integration $\SI{1.36}{\kelvin}$ }   & 0.6467 \pm 0.0153  & 4.5810 \pm 0.5918  & $\SI{1.06 (077)e6}{}$  \\
   \text{Integration $\SI{1.94}{\kelvin}$ }   & 0.4828 \pm 0.0642  & 6.1364 \pm 1.1287  & $\SI{2.27 (069)e9}{}$  \\
  \bottomrule
  \end{tabular}
  \caption{Messwerte, die für die Bestimmung der charakteristischen Relaxationszeit benötigt werden und die charakteristischen Relaxationszeiten.}
  \label{tab:tau}
\end{table}

\noindent
Grafisch aufgetragen finden sich die Funktionen $\tau(T)$ für die einzelnen errechneten Werte in der Abbildung \ref{img:tau}. 
Dabei wurde eine halblogarithmisch Darstellung gewählt. Die Funktionen wurden dabei nur für die Temperaturwerte der einzelnen Messreihen eingezeichnet.

\begin{figure}[ht]
  \centering
  \includegraphics[width=0.7\textwidth]{build/plots/tau_plot.pdf}
  \caption{$\tau(T)$, für die in den einzelnen Auswertungsschritten berechneten Parameter, geplottet. 
          Dabei wurden nur die Werte für $T$ aus den Messreihen genutzt.  }
  \label{img:tau}
\end{figure}
