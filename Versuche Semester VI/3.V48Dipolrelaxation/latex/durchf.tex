\newpage
\section{Aufbau}
	\begin{figure}
        \centering
        \includegraphics[width = \textwidth]{latex/images/Aufbau.jpeg}
        \caption{Ein Bild des verwendeten Versuchsaufbau.}
        \label{fig:Aufb}
    \end{figure}
    \noindent
    In Abbildung \ref{fig:Aufb} ist  eine Aufname des verwendeten Aufbaus. 
    In dem Aufbau auf dem Tisch ist unten rechts ein Spannungsgenerator der bei Bedarf an den Kondensator angeschlossen wird.
    Auf diesem Spannungsgenerator steht ein Heizgerät welches permanent an der Probe angeschlossen ist.
    Auf den beiden Generatoren steht rechts eine Anzeige zum Ablesen des Druckes im Rezipienten und links ein Ampermeter.
    In der Mitte des Aufbaus steht der Rezipient, dieser kann durch einen Kühlfinger auf der Unterseite mit dem dadrunter Dewargefäß verbunden werden.
    Oberhalb des Rezipienten ist ebenfalls das Pirani Vakuummeter zu sehen.
    Auf der linken Seite des Rezipienten steht auf dem Tisch noch ein Gerät zum Ablesen der Temperatur.
    Auf dem Boden links neben dem Tisch steht eine Drehschieber Vakuumpumpe welche den Druck innerhalb des Rezipienten erzeugt.
    Rechts neben dem Tisch stehen zwei Gefäße mit flüssigem Stickstoff, welche zum Befüllen des Dewargefäßes genutzt werden.
     
\section{Durchführung}

    Um die charakteristischen Größen der Dipolrelaxation zu bestimmen, 
    werden zunächst die Dipole Ausgerichtet indem die Probe auf $\SI{50}{\celsius}$ aufgewärmt und dabei ein ein E-Feld mit einer Spannung$\SI{900}{\volt}$ angelegt wird.
    Sobald $\SI{50}{\celsius}$ in der Probe erreicht wurden, kann die Probe nun bei weiterhin angelschaltetem E-Feld abgekühlt werden.
    Dazu wird das Dewargefäß mit flüssigem Stickstoff gefüllt, auf den Ständer gestellt und bis knapp unter der Probe hoch gedreht.
    Sobald die Probe auf $\SI{-50}{\celsius}$ abgekühlt wurde, kann nun das E-Feld abgeschalten werden, indem der Spannungsgenerator runter gedreht und abgeschalten wird.
    Nachdem die Spannung komplett abgefallen ist, wird der Kondensator noch einmal für 10 Minuten an die Erdung des Ampermeters angeschlossen, damit auch wirklich alle Ladungen abgeflossen sind.
    Nun wird der Kondensator am Ampermeter angeschlossen um den Depolarisationsstrom zu messen, und das Heizgerät eingeschalten.
    Für die erste Messreihe wird die Heizrate so eingestellt, dass sich die Temperatur der Probe um $\SI{1.5}{\celsius}$ alle $\SI{30}{\second}$ erhöht.
    Es wird im $\SI{30}{\second}$ Takt die Temperatur und der Strom dokumentiert bis die Temperatur der Probe wieder $\SI{50}{\celsius}$ erreicht. 
    Die zweite Messreihe wird nun vorbereitet in dem bei einer Probentemperatur von mindestens $\SI{50}{\celsius}$, das E-Feld bei einer Spannung von $\SI{900}{\volt}$ für mindestens 15 Minuten eingeschalten bleibt.
    Die restliche Versuchsdurchführung ist nun exakt wie bei der ersten Messreihe, nur mit einer Heizrate von $\SI{4}{\celsius}$ pro Minute.