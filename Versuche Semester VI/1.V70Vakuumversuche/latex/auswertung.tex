\newpage 
\section{Auswertung}

        \begin{table}[ht]
          \centering
          \small
          \caption{Messdaten zur Wien-Robinson-Brücke}
          \label{tab:tab2}
          \begin{tabular}{S [table-format=5.0] S [table-format=1.3]}
           \toprule
           {$\nu \mathbin{\scalebox{1.5} / } \si{\hertz}$} & $\text{Amplitude} \mathbin{\scalebox{1.5} / } \si{\volt}$\\
           \midrule
           20 & 3.2   \\
           100 & 3     \\
           200 & 2     \\
           300 & 1.35  \\
           400 & 0.64  \\
           420 & 0.56  \\
           440 & 0.44  \\
           460 & 0.34  \\
           480 & 0.25  \\
           500 & 0.16  \\
           510 & 0.12  \\
           520 & 0.08  \\
           530 & 0.044 \\
           540 & 0.005 \\
           550 & 0.036 \\
           560 & 0.08  \\
           570 & 0.12  \\ 
           580 & 0.15  \\
           590 & 0.18  \\
           600 & 0.22  \\
           620 & 0.28  \\
           640 & 0.36  \\
           660 & 0.44  \\
           680 & 0.5   \\
           700 & 0.56  \\
           800 & 0.8   \\
           900 & 1.05  \\
          1000 & 1.25  \\
          1500 & 2     \\
          2000 & 2.4   \\
          3000 & 2.7   \\
          4000 & 2.8   \\
          5000 & 2.9   \\
          7500 & 3     \\
         10000 & 2.9   \\
          \bottomrule
          \end{tabular}
        \end{table} 




        \subsection{Fehlerrechnung}
        \noindent
        Die Fortpflanzung von Messungenauigkeiten für mehrere unabhängige Fehler wird durch die Gaußsche Fehlerfortpflanzung
        \begin{equation*}
        \Delta f = \sqrt{\sum_{i \, = \, 1}^{n} \, \left(\frac{\partial f}{\partial x_i} \, \Delta x_i\right)^2}
        \label{fehler}
        \end{equation*}
        beschrieben. Dabei gibt $\Delta x$ die Unsicherheit des arithmetischen Mittelwerts $\bar{x}$ einer Observablen $x$ an:
        
        \begin{equation*}
        \Delta x =  \sqrt{\frac{1}{n(n-1)} \sum_{i \, = \, 1}^{n} \, \left(\bar{x}- x_i\right)^2}.
        \end{equation*}

        \noindent
        Die Zahl $n$ gibt die Anzahl der unabhängigen Messungen an.\\\\
        Die Messwerte, die bei Messungen mit der Turbopumpe aufgenommen wurden, besitzen im Bereich $\SI{1e-8}{\milli\bar}$ bis $\SI{100}{\milli\bar}$ eine Ungenauigkeit von $30$\%.
        Im Bereich von $\SI{100}{\milli\bar}$ bis $\SI{1000}{\milli\bar}$ sind es sogar $50$ \%.\\
        Für die Messungen mit der Drehschieberpumpe sind es für Werte kleiner als $\SI{2e-3}{\milli\bar}$ ein Faktor $2$ vom Messwert.
        Zusätzlich sind von $\SI{2e-3}{\milli\bar}$ bis $\SI{10}{\milli\bar}$ $\pm \SI{120}{\milli\bar}$ und von $\SI{10}{\milli\bar}$ bis $\SI{1200}{\milli\bar}$ $\pm \SI{3.6}{\milli\bar}$.\\\\
        Für die Fehlerfortpflanzung des logarhitmischen Fits wird die aus der Formel \ref*{fehler} hergeleitete Formel 
        \begin{equation*}
          2
        \end{equation*}
        genutzt.

        \subsection*{Volumen}

        \subsection*{Drehschieberpumpe}



















      \newpage
        
        
        \begin{table}
          \begin{center}
            \begin{tabular}{cccccccc}
              \toprule
              $p_\text{Gg} \, / \, \text{mbar}$ & $a \, / \, \text{s}^{-1}$ &       $b$        & $Q \, / \, (\text{mbar \cdot l}/\text{s})$ & $S \, / \, (\text{l}/\text{s})$ &  \\ \midrule
              0,1                &    $0,0050 \pm 0,0002$    & $0,15 \pm 0,02$  &          $ 0,056 \pm 0,005  $           &       $ 0,56 \pm 0,07  $        &  \\
              0,4                &     $0,035 \pm 0,001$     & $0,33 \pm 0,05$  &          $  0,39 \pm 0,03   $           &       $  1,0 \pm 0,1   $        &  \\
              0,6                &     $0,067 \pm 0,002$     & $0,55 \pm 0,07 $ &           $ 0,74 \pm 0,06   $           &        $1,2 \pm 0,2   $         &  \\
              1,0                &    $0,1187 \pm 0,0007$    & $0,96 \pm 0,03 $ &           $  1,3 \pm 0,1   $            &       $  1,3 \pm 0,2   $        &  \\ \bottomrule
              &                           &
            \end{tabular}
            \caption{Die Parameter der Regressionsgeraden in Abbildung \ref{fig:drehLeck} für die Leckratenmessungen der Drehschieberpumpe.}
            \label{tab4}
          \end{center}
        \end{table}

        