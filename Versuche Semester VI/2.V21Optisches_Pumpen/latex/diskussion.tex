\newpage
\section{Diskussion}

	Insgesamt lief der Versuch bis auf die leichten Probleme bei der Durchführung sehr gut.\\
	Es gab ein paar Probleme mit dem Ansteuern der Horizontalfeld-Spule. Diese wurden aber durch einen externen Generator und ein davorgeschaltetes Amperemeter behoben.\\
	Mit Ausnahme des vertikalen Magnetfeldes wurden die Theoriewerte sehr gut getroffen.\\\\
	In dem Versuch wurde das vertikale Magnetfeld zu $\SI{35.25}{\micro\tesla}$ bestimmt. Dies hat zu dem Theoriewert\cite{Erdmag} des vertikalen Magnetfeldes in Deutschland von $\SI{44}{\micro\tesla}$ eine relative Abweichung von $\SI{19.89}{\percent}$.\\
	Dies lässt sich damit erklären, dass das Magnetfeld innerhalb eines Gebäudes mit Stahlbeton gemessen wurde.
	Außerdem konnte die optimale Stromstärke, für die der Effekt des Erdmagnetfeldes minimal ist, auch nur grob abgeschätzt werden.\\
	Die Bestimmung des Kernspins hat sehr gut geklappt. Für die beiden  ergab sich:
	\begin{align*}
		&\text{Isotop 1:} & \text{berechnet: }\num{1.48(05)}\quad &\text{Theorie: }\num{1.5} &\text{relative Abweichung: }\SI{1}{\percent}  \\
		&\text{Isotop 2:} & \text{berechnet: }\num{2.54(05)}\quad &\text{Theorie: }\num{2.5} &\text{relative Abweichung: }\SI{1.7}{\percent}
	\end{align*}
	Hier und bereits in dem Auswertungsteil zum quadratischem Zeemann-Effekt wurde bereits genutzt, dass das Isotop 1 welches einen Kernspin von $\num{1.48}$ besitzt, das $^{87}\text{Rb}$ ist. \\
	Das Isotop 2 mit einem Kernspin von \num{2.54} ist wiederum das $^{85}{\text{Rb}}$.\\
	Die Bestimmung der Anteile der Rubidium Isotope im Rubidium Gas\cite{Rubidium} hat leider nicht perfekt geklappt. 
	Für die beiden Isotope ergab sich:
	\begin{align*}
		&^{85}\text{Rb} & \text{berechnet: }\SI{67.857}{\percent} &\text{\;Theorie: }\SI{72.168}{\percent} &\text{relative Abweichung: }\SI{6.8}{\percent} \\
		&^{87}\text{Rb} & \text{berechnet: }\SI{32.143}{\percent} &\text{\;Theorie: }\SI{27.835}{\percent} &\text{relative Abweichung: }\SI{15.47}{\percent} 
	\end{align*}
	Diese Abweichungen können daran liegen, dass die Berechnung nur mit einem Datenpunkt gemacht wurden. Außerdem können die Peaktiefen nicht sehr genau abgelesen werden.\\
	Alles in allem liefern die Messungen aber sehr gute Ergebnisse. Die Kernspins konnten sehr genau bestimmt werden und die Anteile der Gase hinreichend gut. 
	\newpage

