\newpage
\section{Diskussion}



\begin{figure}[ht]
    \begin{subfigure}{0.46\textwidth}
            \centering
            \includegraphics[width=\textwidth]{build/plots/saug_dreh.pdf}
            \label{img:saug_dreh}
            \caption{Die Saugvermögen aller Messungen für die Drehschieberpumpe, für den Druck aufgetragen.}
    \end{subfigure}
    \hfill
    \begin{subfigure}{0.46\textwidth}
            \centering
            \includegraphics[width=\textwidth]{build/plots/saug_turbo.pdf}
            \label{img:saug_turbo}
            \caption{Die Saugvermögen aller Messungen für die Turbomolekularpumpe, für den Druck aufgetragen.}
    \end{subfigure}
    \label{img:saug}
\end{figure}


\noindent Die Aufnahme der Messwerte lief alles in allem ganz gut. Es gab keine Probleme mit dem Aufbau und die Werte passten auf den ersten Blick zu unseren Erwartungen.\\
Allerdings haben wir vergessen die Enddrücke für die Evakuierungsmessungen abzulesen, weswegen diese abgeschätzt werden mussten.\\
Des Weiteren hat die Turbomolekularpumpe bei der Evakuierungsmessung jedes Mal im selben Messzeitraum ein besseres Vakuum erzeugt. 
Dies lag vermutlich daran, dass bei jedem Pumpen mehr Verunreinigungen entfernt wurde, welche sonst ein virtuelles Leck darstellen würden. 
Allerdings hat dies dann zu einem größeren Fehler des Mittelwerts geführt.\\
Ein zusätzliches Problem war, dass auf Grund des exponentiellen Wachstums und der exponentiellen Abnahme, bei Start einer Messreihe sehr schnell reagiert werden musste, da sich die Werte sehr schnell geändert haben.
Dies führt natürlich auch zu Ungenauigkeiten in den Messungen. Da wir aber trotzdem, auch in diesem Bereich, wenig Varianz in unseren Werten haben, sollte dies zu vernachlässigen sein.\\
Alles in allem liefen die Messungen also gut.\\\\

\noindent
Die Ergebnisse der Drehschieberpumpe sind, wie in Abbildung \ref{img:saug_turbo} zu sehen, recht nah am Theoriewert von $\SI{1.1}{\litre\per\second}$.\\
Die genauen relativen Abweichungen und die dazugehörigen Messwerte sind in Tabelle \ref{tab:abw_dreh} aufgetragen. \\
Dabei werden die Ergebnisse für die letzten beiden Bereiche der Evakuierungsmessung, $S_2 = \SI{}{\litre\per\second}$ und $S_3 = \SI{}{\litre\per\second}$, nicht weiter betrachtet.
Dies liegt daran, dass der Theoriewert nur für den optimalen Wirkungsbereich der Pumpe gilt.\\
Die Bereiche für die diese Werte berechnet wurden, sind davon schon so weit entfernt, dass dort andere Saugvermögen gelten. \\
\begin{table}[H]
    \centering
    \small
    \label{tab:abw_dreh}
    \begin{tabular}{S [table-format=5.0]  c c}
     \toprule
     {Verfahren} & $\text{L} \mathbin{\scalebox{1.5} / } \si{\litre\per\second}$ & $\text{relative Abweichung} \mathbin{\scalebox{1.5} / } \si{\percent}$ \\
     \midrule
     \text{Evakuierung}                 &  1.0532 \pm 0.10562      & 4.2580 \pm 9.60221 \\
     \text{Leck $\SI{0.4}{\milli\bar}$}  & 0.7310 \pm 0.09358        & 33.5432 \pm 8.50711 \\
     \text{Leck $\SI{0.4}{\milli\bar}$}  & 1.2926 \pm 0.13003      &  -17.5067 \pm 11.82046 \\
     \text{Leck $\SI{0.4}{\milli\bar}$}  & 1.1492 \pm 0.11621      & -8.9793 \pm 10.90703 \\
     \text{Leck $\SI{0.4}{\milli\bar}$}  & 1.1988 \pm 0.11998      & -4.4690 \pm 10.56470 \\
    \bottomrule
    \end{tabular}
\end{table} 
\noindent
An Tabelle \ref{tab:abw_dreh} lässt sich erkennen, dass die Drehschieberpumpenmessung insgesamt nah an den Theoriewert gekommen ist.\\\\

\noindent Anders ist es hingegen bei der Turbomolekularpumpe. \\
Es lässt sich schon in Abbildung \ref{img:saug_turbo} erkennen, dass kein bestimmtes Saugvermögen auch nur nah an den Theoriewert von $S_{theo} = \SI{77}{\litre\per\second}$ kommt.
Und das, obwohl es bei der Messung keine erkennbaren Probleme gab.\\ 
Es gab keine besonders große Varianz unter den einzelnen Messreihen. 
Die errechneten Saugvermögen weichen stark vom Theoriewert ab,
schwanken aber trotzdem alle um $\approx \SI{5}{\litre\per\second}$.\\
Dies zeigt auf jeden Fall, dass der Rechenvorgang an sich nicht falsch ist. 
Des Weiteren wurde die Auswertung der Drehschieberpumpe mit denselben Funktionen durchgeführt.\\
Es könnte natürlich ein Größenordnungsfehler im Code sein. So etwas konnten wir aber nicht finden.\\
Woran diese Ergebnisse liegen können wir uns also nicht erklären.\\\\
Die Ergebnisse und ihre relative Abweichung vom Theoriewert sind auch noch einmal in Tabelle \ref{tab:abw_turbo} aufgetragen.\\
Auch hier werden wieder die späteren Bereiche der Evakuierungsmessung außen vor gelassen.\\
In der Tabelle zeigt sich auch noch einmal wie groß die Abweichungen wirklich sind. 
\begin{table}[H]
    \centering
    \small
    \label{tab:abw_dreh}
    \begin{tabular}{S [table-format=5.0]  c c}
     \toprule
     {Verfahren} & $\text{L} \mathbin{\scalebox{1.5} / } \si{\litre\per\second}$ & $\text{relative Abweichung} \mathbin{\scalebox{1.5} / } \si{\percent}$ \\
     \midrule
     \text{Evakuierung Ventil }                  &  5.4995 \pm 1.90637      & 92.8578 \pm 2.47580 \\
     \text{Evakuierung Pumpe }                   &  5.6086 \pm 1.21505      & 92.7161 \pm 1.57799 \\
     \text{Leck $\SI{1e-4}{\milli\bar}$}         & 5.1067 \pm 0.52772        & 93.3679 \pm 0.68535 \\
     \text{Leck $\SI{2e-4}{\milli\bar}$}          & 8.3083 \pm 0.93574      &  89.2100 \pm 1.21525 \\
     \text{Leck $\SI{7e-4}{\milli\bar}$}         & 4.4886 \pm 0.46088      & 94.1707 \pm 0.59855 \\
     \text{Leck $\SI{5e-4}{\milli\bar}$}         & 4.2640 \pm 0.43101      &  94.4624 \pm 0.55976 \\
    \bottomrule
    \end{tabular}
\end{table} 
\noindent
Es bleibt noch der zwei Messreihen der Evakuierungsmessung mit der Turbomolekularpumpe, welche zur Bestimmung des Leitwerts genutzt wurde.\\
Da wir keine Theoriewerte besitzen fällt es schwer zu vergleichen. Allerdings ist an den Messwerten direkt ersichtlich, dass es einen Unterschied zwischen den beiden Messstellen gibt.\\
Die errechneten Leitwerte zeigen dies auch und geben nun 
alles in allem lässt sich also sagen, dass die Ergebnisse der Drehschieberpumpe sehr gut und die der Turbomolekularpumpe sehr schlecht sind.
