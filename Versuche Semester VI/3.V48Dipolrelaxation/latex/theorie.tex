\section{Zielsetzung}
Dipole durch dotierung\\
Anregungsenergie\\
Relaxationszeit

\section{Theorie}
	\subsection{Dipole in dotierten Ionenkristallen}
		regelmäßiges Gitter aus Ionen\\
		insgesamt elektrisch neutral\\
		dotierung führt zu dipol\\
		bei raumtemperatur in summe kein dipolmoment\\

	\subsection{Depolarisationseffekte}
		Anfangsbedingung: Dipole sind in eine Richtung ausgerichtet und eingefroren\\
		Beim aufwärmen der Probe wird der Strom gemessen\\
		Reorientierung der Dipole\\
		unter 500C Leerstellendiffusion, dazu materialspezifische Aktivierungsenergie $W$\\
		Energie im Kristall durch Blotzmann-Statistik $\text{~exp}\left(\frac{-W}{k_\text{B}T}\right)$\\
		\begin{equation}
			\tau(T) = \tau_0 \text{exp}\left(\frac{-W}{k_\text{B}T}\right)
		\end{equation}
		$\tau_0 = \tau(\infty)$

	\subsection{Polarisationsansatz}
		Depolarisationsstrom:	
		\begin{equation}
			I(T) = - \frac{\text{d}P(t)}{\text{d}t}
		\end{equation}
		Polarisationsrate:
		\begin{equation}
			\frac{\t{d} P(t)}{\t{d} t} = \frac{P(t)}{\tau(T)}
			\label{eqn:diff}
		\end{equation}
		ergibt:
		\begin{equation}
			I(T) = \frac{P(t)}{\tau(T)}
		\end{equation}
		Seperation der Variabeln von \ref{eqn:diff}:
		\begin{equation}
			P(t) = P_0 \t{exp}\left(- \frac{t}{\tau(T)} \right)
		\end{equation}
		Es egibt sich
		\begin{equation}
			I(T) = \frac{P_0}{\tau(T)} \t{exp}\left(-\frac{t}{\tau(T)}\right)
		\end{equation}
		Hier gibt $t$ die Zeit an, die benötigt wurde um $T$ zu erreichen, sie lässt sich auch als Integral schreiben:
		\begin{equation}
			I(T) = \frac{P_0}{\tau(T)} \t{exp}\left(-\int_0^t\frac{\t{d}t}{\tau(T)}\right)
			\end{equation}
		mittels einer konstanten Heizrate
		\begin{equation}
			b := \DIF{T}{t} = const
		\end{equation}
		lässt sich der Depolarisationsstrom als 
		\begin{equation}
			I(T) = \frac{P_0}{\tau(T)} \t{exp}\left(\frac{-1}{b\tau_0}\int_{T_0}^T\frac{\t{d}T'}{\tau(T')}\right)
		\end{equation}
		ausdrücken.

	\subsection{Stromdichtenansatz}
		
		mittlere Polarisation:
		\begin{equation}
			\bar{P}(T) = \frac{N}{N_V}\frac{p^2E}{3k_\t{B}T}
		\end{equation}	
		mit dem Dipolmoment $p$, der elektrischen Feldstärke $E$, der Temperatur $T$ und der Dipoldichte $N_V$.
		Die Geschwindigkeit der relaxierenden Dipole ist
		\begin{equation}
			\DIF{N(T)}{t} =- \frac{N}{\tau(T)}
		\end{equation}
		Analog zum vorherigen Kapitel egibt sich die Lösung der Differentialgelich zu:
		\begin{equation}
			N = N_\t{P} \t{exp}\left( \frac{-1}{b}\int_{T_0}^T \frac{\t{d}T'}{\tau(T')}\right)
		\end{equation}
		Weiterhin gilt
		\begin{align}
			I(T) = \bar{P}(T)\DIF{N}{t}& &\t{und} & & I(T) = -\bar{P}(T) \frac{N}{\tau(T)}
		\end{align}
		Zusammensetzten aller dieser Terme egibt dann
		\begin{equation}
			I(T) =\frac{p^2E}{3k_\t{B}T}\frac{N_\t{P}}{\tau_0}\t{exp}\left(\frac{-1}{b\tau_0}\int_{T_0}^T\frac{\t{d}T'}{\tau(T')}\right)\t{exp}\left(-\frac{W}{k_\t{B}T}\right)
			\label{eqn:long}
		\end{equation}		

	\subsection{Berechnung der Aktivierungsenergie $W$}
		
		\subsubsection{Bestimmung mithilfe des Maximums}

				Wird angenommen, dass die Aktivierungsenergie $W$ groß gegenüber der Energie $k_\t{B}T$ und der Temperaturdifferenz $T-T_0$, so wird das Integral in Gleichung\ref{eqn:long} 
				\begin{equation}
					\int_{T_0}^T\frac{\t{d}T'}{\tau(T')} \approx 0
				\end{equation}
				Somit ergibt sich dich der Strom dann zu
				\begin{equation}
					\frac{P^2E}{3k_\t{B}T}\frac{N_\t{P}}{\tau_0} \t{exp}\left(-\frac{W}{k_\t{B}T}\right)
				\end{equation}	
				Mittes des Logarithmus entsteht hieraus eine Geradengleichung
				\begin{equation}
					\t{ln}(I(T)) = \left(\frac{P^2EN_\t{P}}{3k_\t{B}T\tau_0}\right) - \frac{W}{k_\t{B}} \frac{1}{T}
				\end{equation}
				Die Steigung dieser Geraden ist also $\frac{W}{k_\t{B}}$ oder
				\begin{equation}
					W = m \cdot k_\t{B}
				\end{equation}
				mit der Steigung $m$
