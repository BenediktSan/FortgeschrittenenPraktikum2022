\section{Zielsetzung}
    Ziel dieses Versuches ist es die Lebensdauer kosmischer Myonen zu bestimmen und die verwendeten Schaltungen und Bauteile genauer zu verstehen.

    \section{Enstehung und Zerfall kosmischer Myonen}
        Myonen entstehen in der Atmossphäre in sogenannten Luftschauern wenn hochenergetische kosmische Teilchen auf die Erdatmossphäre Treffen.
        In unterschiedlichen Reaktionen entstehen unteranderem geladene Pionen welche wiederum in Myonen zerfallen:
        \begin{align}
            &\pi^+ \rightarrow \mu^+ + \nu_\mu && \text{und} && \pi^- \rightarrow \mu^- + \bar{\nu}_\mu \, .
        \end{align}
        Diese so enstandenen Myonen sind so hochenergetisch, dass sie nach dem Pionen Zerfall mit annähernd Lichtgeschwindigkeit in Richtung Erde fliegen.
        Nur aufgrund der relativistischen Geschwindigkeiten ist es den Myonen möglich auf der Erdoberfläche anzukommen. 
        Denn im Vergleich zum Elektron, sind Myonen instabil und besitzten eine ca. 207 mal größere Ruhemasse. 
        Myonen zerfallen nach folgener Gleichung in Elektronen und Neutrinos:
        \begin{align}
            \mu^- \rightarrow e^- + \bar{\nu}_e + \nu_\mu && \text{und} &&  \mu^+ \rightarrow e^+ + \nu_e + \bar{\nu}_\mu \, . 
        \end{align} 

    \subsection{Lebensdauer instabiler Teilchen}

        Der Zerfall eines Elementarteilchen ist ein statischer Prozess, jedes Teilchen hat die selbe Wahrscheinlichkeit zu Zerfallen.
        Die Wahrscheinlichkeit d$W$, dass ein Teilchen zerfällt ist also proportional zu der Zeit d$t$, es folgt:
        \begin{equation}
            \t{d}W = \lambda\t{d}t \, .
        \end{equation}
        Weiterhin gilt somit auch für den Zerfall vieler Teilchen:
        \begin{equation}
            \t{d}N = -N\t{d}W = -\lambda N \t{d}t \, ,
        \end{equation}
       hier ist d$N$ die Anzahl an Zerfällen in dem Zeitraum d$t$, wenn $N$ Teilchen betrachtet wurden.
       Integrieren dieser Gleichung ergibt das Zerfallsgesetz:
       \begin{equation}
           N(t) = N_0 \cdot \t{exp}(-\lambda t)\, .
       \end{equation}
       Hier ist $\lambda$ die Zerfallskonstante, $t$ die Zeit und $N_0$ die Gesamtzahl der betrachteten Teilchen. 
       Die Zeit nachdem die noch übrig gebliebenen Teilchen $N(t)$ auf $\frac{N_0}{e}$ abgefallen sind, wird als Lebensdauer $\tau$ eines Teilchens beizeichent und berechnet sich zu $\tau = \frac{1}{\lambda}$.





