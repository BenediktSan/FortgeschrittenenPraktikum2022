\section{Aufbau}

\section{Durchführung}
	Zunächst wird die Funktionsfähigkeit der Analge überprüft und vorbereitet. 
	Dazu wird getestet ob die Drehschieberpumpe innerhalb von maximal 10 Minuten in der lage ist einen Enddruck $P_\text{E}$ von 0,03 mbar bis 0,05 mbar zu erzeugen. 
	Ist dem nicht so, muss die Anlage auf undichte Stellen überprüft werden.
	Weiterhin wird dann mit dem bereits vorhandenen Vorvakuum, die Turbopumpe eingeschalten. 
	Um Wasseranlagen zu entfernen un ... vorzubeugen wird die Anlage auch einmal mit einem Heißluftfön erhitzt.
	Die Turbopumpe sollte einen Druckb von $2 \cdot 10^{-5}$ mbar bis $8 \cdot 1o^{-5}$ mbar erzeugen können.

	\subsection{Messungen Zur Drehschieberpumpe}
		Sobald bestätigt wurde, dass der Pumpstand ausreichend dicht ist, können Evakueirungskurven aufgenommen und Leckratenmessungen durchgeführt werden.

		\subsubsection{Evakuierungskurve}

			Zunächste kommt, dass "abschiebern" der Turbopumpe, dazu wird V1 und V5 geschlossen und V2 sowie V4 geöffnet.
			Dann wird bei bereits laufender Drehschieberpumpe der Rezipient belüftet indem für ca. 5 Sekunden D1 und V3 geöffnet wird. 
			Sobald der Rezipient wieder dicht ist, kann dann der Druckabfall als Funktion der Zeit vermssen werden, dazu bieten sich Messzeiten von 180-300 Sekunden an.
			Bei dieser Messung sollte eine Enddruck von $P_\text{E}$ von zwischen 0,04 mbar und 0,06 mbar erreicht werden.
			Diese Messung wird dann 5-mal wiederholt.

		\subsubsection{Saugvermögen}

			Um das Saugvermögen $S$ der Pumpe zu bestimmen, wird eine Leckratenmessung durchgeführt. 
			Dazu wird mittels des Nadelventils ein Gleichgewichtsdruck $p_\text{g}$ eingestellt und dann bei weithin offenem Dosierventil die Pumpe vom System abgeschoben. 
			Den darauf folgenen Druckanstieg wird dann als Funktion der Zeit gemessen. 
			Diese Messung wird mit 4 Gleichgeichtsdrücken $p_\text{g}$ = 0,1;0,4;0,8 und 1,0 mbar und jeweils 3 Messreihen durchgeführt.


  
