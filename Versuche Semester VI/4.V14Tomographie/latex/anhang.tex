\newpage
\section{Anhang}




\subsection{Tabellen}

\subsubsection{Messdaten}

%    \begin{table}[H]
%        \centering
%        \begin{tabular}{S [table-format=2.3] S [table-format=2.3] S [table-format=2.3] c }
%            \toprule
%            \multicolumn{1}{p{3.2cm}}{\centering$\text{Druck Messreihe 1 } $\\$ \text{in }\si{\milli\bar}$} &
%            \multicolumn{1}{p{3.2cm}}{\centering$\text{Druck Messreihe 2 } $\\$ \text{in }\si{\milli\bar}$} &
%            \multicolumn{1}{p{3.2cm}}{\centering$\text{Druck Messreihe 3 } $\\$ \text{in }\si{\milli\bar}$} &
%            \multicolumn{1}{p{3.2cm}}{\centering$\text{Druck gemittelt } $\\$ \text{in }\si{\milli\bar}$} \\
%            \midrule
%
%            \end{tabular}
%            \caption{Messwerte der Leckratenmessung für den Gleichgewichtsdruck $\SI{0.4}{\milli\bar}$ mit der Drehschieberpumpe.}
%            \label{tab:dreh_leck_1}
%    \end{table}

\subsection{Aufbau}

\begin{figure}[h]
    \centering
    \includegraphics[width=0.7\textwidth]{latex/images/Aufbau_real.jpeg}
    \caption{Ein Foto des Versuchsaufbau.}
\end{figure}


\subsection{Messwertfotos}

%\begin{figure}[h]
%    \centering
%    \includegraphics[width=0.7\textwidth]{latex/images/Messwerte_8.jpeg}
%    \caption{Die Messwerte des Vakuumversuchs}
%\end{figure}



