\newpage 
\section{Auswertung}

  \subsection{Spektrum der $^{137}\t{Cs}$-Quelle}
    Das Spektrum des Strahlers wurde in diesem Versuch mit einem NaI-Szintillationsdetektor aufgenommen. 
    In Abbildung \ref{fig:Spektrum} sind die Ergebnisse der Aufnahme in einem Histogramm dargestellt. Auf der $x$-Achse sind die Channels des Multichannelanalyser abgebildet.
    \begin{figure}[H]
      \centering  
      \includegraphics[width = 0.55\linewidth]{build/plots/spektrum.pdf}
      \caption{Das Aufgenommene Spektrum der Cs-Quelle.}
      \label{fig:Spektrum}
    \end{figure}

    \noindent In der Abbildung \ref{fig:Spektrum} sind nur die Anzahl der Ereignisse gegen die Channel aufgetragen. 
    Die Energie der Strahlung wird von dem Multichannelanalyser nicht vermessen. 
    Jedoch lässt sich aus der Literatur \cite{leifi} entnehmen, dass der Photopeak des $\ce{^{137}Cs}$-Strahlers bei $E_P = \SI{662}{\kilo\electronvolt}$ zu finden ist.
    Dieser Photpeak der Strahlungsquelle ist in Abbildung \ref{fig:Spektrum} ca. bei Channel 65 zu finden. 
    Bei Channel 43 befindet sich die Comptonkante mit $E_C = \SI{478}{\kilo\electronvolt}$ und bei Channel 21 der Rückstreupeak.

  \subsection{Würfel 1}
    Zunächst wird die Nullrate bestimmt, welche durch die Aluminiumhülle dringt und alle folgenden Würfel umgibt. 
    Die in dieser Messreihe bestimmten Intensitäten des Photopeaks wird daher als Nullrate $N_0$ für die folgenden Messungen benutzt.
    Hier wird dazwischen unterschieden wie der Strahl durch den Block dringt.
    Dies geschieht entweder gerade, also senkrecht auf einer Seite, oder auf einer Haupt- oder Nebendiagonalen.\\
    Da es sich hier um ein Zählexperiment handelt, ist der Fehler Poissonverteilt mit $\sigma_N = \sqrt{N}$.
    Auswertung der Net-Area um den Photopeak ergibt für die Nullrate die Werte aus 
    Tabelle \ref{tab:w1}.
    Nullrate beschreibt hier und in der folgenden Auswertung das Integral des Photopeaks. 
    Also die Summe der Events welche durch den Photopeak der $^{137}\text{Cs}$-Quelle entstehen.
    Die Bezeichnung einer Rate ist im folgenden nur korrekt, da alle Messreihen über die selber Zeit von $t = \SI{240}{\second}$ aufgenommen wurden und diese somit nicht relevant ist.
    Sämtliche Größen, ob Raten, Anzahlen oder Intensitäten sind in dieser Auswertung somit identisch und einheitenlos.
    \begin{table}[H]
    \centering
     \caption{Die gemessene Anzahl der Ereignisse der Messung des leeren Würfel 1, der nur aus der Aluminiumhülle besteht.}
     \label{tab:w1}
     \begin{tabular}{c S[table-format=4.0] @{${}\pm{ }$} S[table-format=3.0]} 
         \toprule
         {Strahlengang} & \multicolumn{2}{c}{$N$}  \\
         \midrule
         Gerade                 &  3704 & 190  \\ 
         Hauptdiagonale         &  3636 & 190  \\ 
         Nebendiagonale         &  3796 & 190  \\ 
         \bottomrule 
     \end{tabular}
    \end{table}
  
  \subsection{Würfel 2}
    Da der Würfel 2 homogen ist, und aufgrund seiner geringen Dichte große Statistik erzeugt, wurden für diesen Block nur 2 Messreihen durchgeführt.
    Es wurde eine Net-Area von $N_G=30323$ für die Gerade und $N_D=29521$ für die Diagonale gemessen.
    Damit berechnen sich die Absorptionskoeffizienten zu $\mu_G = \SI{0.0699(00026)}{1\per\centi\metre}$ und $\mu_D=\SI{0.0491(00018)}{1\per\centi\metre}$.
    Dazu wurde die Formel
    \begin{equation}
      \mu = \frac{\t{ln}\left(\frac{N_0}{N}\right)}{d} \, ,
    \end{equation}
    mit der Nullrate $N_0$ der Net-Area $N$ und der Länge $d$, welche die Strahlung innerhalb des Aluminiumgehäuses durchquert, benutzt
    Im Mittel ergibt sich für Block 2 ein Absorptionskoeffizienten von: 
    \begin{equation}
      \mu_\t{Würfel 2} = \SI{0.0595(00016)}{1\per\centi\metre}.
    \end{equation}

  \subsection{Würfel 3}
    Würfel 3 ist auch homogen. Jedoch ließ seine hohe Dichte bereits auf einen hohen Absorptionskoeffizienten schließen. 
    Um gute Werte sicherzustellen wurden für diesen Block mehr Messreihen durchgeführt:
    \begin{table}[H]
     \centering
     \caption{Die Messwerte und daraus errechneten Werte der Messung des Würfel 3.}
     \label{tab:w2}
     \begin{tabular}{c S[table-format=4.0] S[table-format=1.3] @{${}\pm{}$} S[table-format=1.3]}
       \toprule
       {Projektion} &\multicolumn{1}{c}{$N$}  & \multicolumn{2}{c}{$\mu \mathbin{/} \left(\si{\per\centi\metre}\right)$} \\
       \midrule
       Gerade & 1330 & 1.112 & 0.009 \\
       Gerade & 1375 & 1.101 & 0.009 \\
       Gerade & 1366 & 1.103 & 0.009 \\
       Diagonale & 614 & 0.962 & 0.010 \\
       Diagonale & 568 & 0.980 & 0.010 \\
       Nebendiagonale & 2258 & 0.998 & 0.008 \\
       \bottomrule  
     \end{tabular}
    \end{table} 
    \noindent
    Im Mittel ergibt dies einen Absorptionskoeffizienten von $\mu_\t{Würfel 3} = \SI{1.045(0004)}{1\per\centi\metre}$.

  \subsection{Würfel 4}
    Würfel 4 besteht insgesamt aus 27 kleineren jeweils homogenen Würfeln.
    In diesem Versuch wird eine horizontale Ebene mit 9 Würfeln mittels 12 Projektionen ausgewertet.
    Die Projektionen und die zugehörige Matrix $A$ sind im Abschnitt \ref{sec:proj} erklärt und dargestellt.
    Mittels des Vektors an skalierten Intensitäten $\vec{I}$, ergeben sich nach der Formel des Kleinste-Quadrate Fittes
    \begin{equation}
      \vec{\mu} = \left( A^T \cdot A \right)^{-1} \cdot A^T \vec{I}
    \end{equation}
    die einzelnen Absorptionskoeffizienten für die 9 kleinen Blöcke.
    In dieser Transformation, transformieren sich die Unsicherheiten nach der Formel
    \begin{equation}
      V[\vec{\mu}] = \left( A^T \cdot A \right)^{-1} \cdot A^T \, V[\vec{I}] \, A \cdot \left( \left( A^T \cdot A \right)^{-1} \right)^T \,.
    \end{equation}
    $V[\vec{\mu}]$ ist hier eine Kovarianzmatrix, welche die Unsicherheiten der einzelnen Absorptionskoeffizienten $\mu_i$ auf der Hauptdiagonalen hat.
    In Tabelle  \ref{tab:w4} sind nun die Projektionen mit den gemessenen Events und den nach
    \begin{equation}
      I_i = \t{ln}\left( \frac{N_{G/D/N}}{N_i} \right)
    \end{equation}
    berechneten Intensitäten aufgetragen. $N_{G/D/N}$ steht hier für die Nullrate, je nachdem ob der Strahl gerade ($G$) durch den Block geht oder eine Haupt ($D$)-, oder Nebendiagonale ($N$) durchstrahlt.
    \begin{table}[H]
    \centering
    \caption{Die gemessenen Anzahlen der Ereignisse unter dem Photopeak und die daraus errechneten Werte $I_i$ von der Messung des Würfel 4.}
    \label{tab:w4}
    \begin{tabular}{c S[table-format=4.0] S[table-format=2.3] @{${}\pm{}$} S[table-format=1.3] }
      \toprule
      {Projektion} & \multicolumn{1}{c}{$N$} & \multicolumn{2}{c}{$I \mathbin{/} \left(\si{\per\second}\right)$} \\
      \midrule
      $I_1 $   &  10972 & 1.226 & 0.010 \\
      $I_2 $   &  11123 & 1.212 & 0.010\\
      $I_3 $   &  10915 & 1.231 & 0.010\\
      $I_4 $   &  29942 & 0.222 & 0.007\\
      $I_5 $   &   1475 & 3.233 & 0.026\\
      $I_6 $   &  29503 & 11.968& 0.029 \\
      $I_7 $   &   6745 & 1.712 & 0.013\\
      $I_8 $   &  58814 & 1.861 & 0.014\\
      $I_9 $   &   7040 & 1.670 & 0.012\\
      $I_{10}$ &   7959 & 1.547 & 0.012\\
      $I_{11}$ &   7281 & 1.636 & 0.012\\
      $I_{12}$ &   9338 & 1.387 & 0.011\\
      \bottomrule
    \end{tabular}
  \end{table}
  \noindent
  Die mit diesen Werten berechneten Absorptionskoeffizienten sind in Tabelle \ref{tab:mu4} zu finden.
  Da diese Ergebnisse in Teilen unphysikalisch sind, da teilweise negativ, werden erst in der Diskussion physikalische Interpretationen der Ergebnisse durchgeführt.

  \begin{table}[H]
    \centering
    \caption{Die ermittelten Werte für die Absorptionskoeffizienten der verschiedenen kleineren Würfel.}
    \label{tab:mu4}
    \begin{tabular}{c S[table-format=2.4] @{${}\pm{ }$} S[table-format=1.4] }
      \toprule
      {Würfel} & \multicolumn{2}{c}{$\mu \mathbin{/} \left(\si{\per\centi\metre}\right)$}  \\
      \midrule
      1 & -1.8506& 0.0088 \\
      2 & 0.3600 & 0.0080 \\
      3 & 4.0970 & 0.0119 \\
      4 & 1.2653 & 0.0064 \\
      5 & -0.4721& 0.0090 \\
      6 & 1.2207 & 0.0063 \\
      7 & -1.7298& 0.0089 \\
      8 & 0.2287 & 0.0079 \\
      9 & 4.1127 & 0.0118 \\
      \bottomrule
    \end{tabular}
  \end{table}