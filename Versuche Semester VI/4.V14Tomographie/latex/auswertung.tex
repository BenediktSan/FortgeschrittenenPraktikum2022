\newpage 
\section{Auswertung}

        \subsection{Fehlerrechnung}
        \noindent
        Die Fortpflanzung von Messungenauigkeiten für mehrere unabhängige Fehler wird durch die Gaußsche Fehlerfortpflanzung
        \begin{equation*}
        \Delta f = \sqrt{\sum_{i \, = \, 1}^{n} \, \left(\frac{\partial f}{\partial x_i} \, \Delta x_i\right)^2}
        \label{fehler}
        \end{equation*}
        beschrieben. Dabei gibt $\Delta x$ die Unsicherheit des arithmetischen Mittelwerts $\bar{x}$ einer Observablen $x$ an:
        
        \begin{equation*}
        \Delta x =  \sqrt{\frac{1}{n(n-1)} \sum_{i \, = \, 1}^{n} \, \left(\bar{x}- x_i\right)^2}.
        \end{equation*}

        \noindent
        Die Zahl $n$ gibt die Anzahl der unabhängigen Messungen an.\\\\
        Die Messwerte, die bei Messungen mit der Turbopumpe aufgenommen wurden, besitzen im Bereich $\SI{1e-8}{\milli\bar}$ bis $\SI{100}{\milli\bar}$ eine Ungenauigkeit von $30$\%.
        Im Bereich von $\SI{100}{\milli\bar}$ bis $\SI{1000}{\milli\bar}$ sind es sogar $50$ \%.\\
        Für die Messungen mit der Drehschieberpumpe sind es für Werte kleiner als $\SI{2e-3}{\milli\bar}$ ein Faktor $2$ vom Messwert.
        Zusätzlich sind es von $\SI{2e-3}{\milli\bar}$ bis $\SI{10}{\milli\bar}$ $\pm\; \SI{120}{\milli\bar}$ 
        und von $\SI{10}{\milli\bar}$ bis $\SI{1200}{\milli\bar}$ $\pm \;\SI{3.6}{\milli\bar}$.\\\\

        genutzt. Des Weiteren wird für die relative Abweichung berechneter Werte vom Theoriewert die Formel 
        \begin{equation*}
          \increment x = \frac{x - x_{theo}}{x_{theo}}
        \end{equation*}
        genutzt.
